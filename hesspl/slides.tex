% Slides for the talk at HessPL on September 11, 2014.
% http://ps-mr.github.io/hesspl-2014/

\documentclass{beamer}
\usetheme{default}
\setbeamertemplate{navigation symbols}{}
\setbeamertemplate{footline}[frame number]

\usepackage[utf8x]{inputenc}
\usepackage{verbatim}

\title{A Language for the Specification and Efficient Implementation
  of Type Systems}
\author{Pascal Wittmann}
\institute{TU Darmstadt}
\date{September 11, 2014}

\begin{document}

\begin{frame}[plain]
  \titlepage{}
\end{frame}

\begin{frame}
  \frametitle{Motivation}
  \begin{itemize}
  \item Type systems provide
    \begin{itemize}
    \item static approximation of programs semantics
    \item means to establish and enforce abstraction barriers
    \item documentation that is always correct
    \end{itemize}
  \item Domains specific languages benefit from specialized type
    systems
  \item Gap between formal definitions of type systems and their
    implementations
  \end{itemize}
\end{frame}

\begin{frame}
  \frametitle{Research Problem}
  \begin{itemize}
  \item Design of a declarative specification language that
    \begin{itemize}
    \item is close to text-book formalisms
    \item makes it easy to use existing programming language definitions
    \end{itemize}
  \item Generate first-order formula representations of specifications
  \item Develop a type checker generator which
    \begin{itemize}
    \item is constraint-based
    \item can cope with non-syntax directed rules
    \item takes advantage of facts proven by automated theorem provers
    \end{itemize}
  \end{itemize}
\end{frame}

\begin{frame}
  \frametitle{References}
  \bibliographystyle{amsalpha}
  \bibliography{../report/bibliography.bib}
\end{frame}

\end{document}