\chapter{Introduction}
\section{Motiviation}
Type systems ensure that programs are well-behaved. In other words,
they try to ensure that programs have meaning in the sense of the
semantics of the programming language. The type systems we focus on
are static type systems and can be thought of as a static
approximation of the program semantics. Besides ensuring that programs
are well-behaved, type systems are means to establish abstractions and
to enforce adherence to these abstractions. They can also serve as
documentation that cannot become obsolete and give hints for
optimization to the compiler. All in all type systems can help to
develop software more efficiently
(cf.~\cite{Petersen:2014:ECS:2597008.2597152}
and~\cite{Mayer:2012:ESI:2384616.2384666}).

Type systems are useful tools, particularly if they fit to the
programming language and the application scenario. Ensuring this,
e.g.\ in the context of a \gls{dsl}, it makes sense to adapt and
modify existing type systems or create new specialized type
systems. Those specializations can lead to better error messages, more
expressive type systems and the detection of more errors
\cite{Thiemann02programmabletype}. Currently there are, to our best
knowledge, no established tools that generate type checkers from a
declarative specification. Such generators could make the development
of type checkers faster and less error prone. Those generators would
fit well into the language development workbench besides the long
established lexer and parser generators.

The goal of this thesis is to close that gap in the tool chain. A
declarative specification language is developed, in which type systems
can be represented very close to the standard formalisms. Two things
are generated from those specifications. First, a representation of
the type system as first-order formulas, intended to support proving
properties of the type system. Second, an efficient type checker that
can be integrated in existing tool chains.
\section{Contributions}
The main contributions of this thesis are:
\begin{enumerate}
\item A declarative specification language for static type systems, in
  which typing rules can be written in the style of natural deduction
  and which builds \gls{sdf} definitions of programming languages.
\item The transformation of type system specifications into equivalent
  first-order formulas in the \gls{tptp} format, suitable to be passed to
  automated theorem provers.
\item A generator that creates type checkers from type system
  specifications and which is designed for performance.
\end{enumerate}

\section{Structure}
\todo[inline]{Describe how the thesis is structured.}

%%% Local Variables:
%%% mode: latex
%%% TeX-master: "../report"
%%% End: