\chapter{Introduction}
\section{Motiviation}
\todo[inline]{Somehow mention that we are always talking about static
  type systems}
\todo[inline]{Mention studies on type systems, e.g. by Hanenberg}

Type systems try to solve the problem of ensuring that programs have
meaning. They assign types to expressions and have rules that state
which expressions of which types can be combined. A type checker
ensures that the expressions of a program meet those rules, if those
rules are met, the program is called well-typed. Essentially a type
system ensures that a program has meaning, by approximating its
semantics. They also can be used to establish abstractions and to
enforce the adherence to these abstractions.

Although type systems are a useful tool increase software quality,
there is a lack of tools that help to create and modify type
checkers. Almost all type checkers are implemented by hand and are far
away from their formal models. In addition it is non-trivial to write
or even extend a type checker. Therefore current tools are not suited
for quick prototyping of type system features for new or existing
languages. This is also true for the development of domain specific
languages (DSL). In that sense type systems are underprivileged in
comparison to parsers. Parser generators are widely spread and
accepted tools.

The goal of this thesis is to close that gap in the tool chain. A
declarative specification language is developed, in which type systems
can be represented very close to the standard formalisms. Two things
are generated from those specifications. First, a representation of
the type system as first-order formulas, intended to support proving
properties of the type system. Second, an efficient type checker that
can be integrated in existing tool chains.
\section{Contributions}
The main contributions of this thesis are:
\begin{enumerate}
\item A declarative specification language for static type systems, in
  which typing rules can be written in the style of natural deduction
  and which builds SDF definitions of programming languages.
\item The transformation of type system specifications into equivalent
  first-order formulas in the TPTP format, suitable to be passed to
  automated theorem provers.
\item A generator that creates type checkers from type system
  specifications and which is designed for performance.
\end{enumerate}

\section{Structure}
\todo[inline]{Describe how the thesis is structured.}

%%% Local Variables:
%%% mode: latex
%%% TeX-master: "../thesis"
%%% End: