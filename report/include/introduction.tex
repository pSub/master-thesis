\chapter{Introduction}
This chapter motivates the thesis, summarizes the contributions and
gives and overview of the structure of the thesis.

\section{Motiviation}
Type systems ensure that programs are well-behaved. In other words,
they ensure that programs have meaning in the sense of the semantics
of the programming language. The type systems we focus on are static
type systems and can be thought of as a static approximation of the
program semantics. Besides ensuring that programs are well-behaved,
type systems are means to establish abstractions and to enforce
adherence to these abstractions. Type systems can provide explicit
type annotations that serve as program documentation, which cannot
become obsolete as type annotations are verified by the type
system. During type checking the program under consideration can also
be annotated with optimization hints for the compiler. All in all type
systems can help to develop software more efficiently
(cf.~\cite{Petersen:2014:ECS:2597008.2597152}
and~\cite{Mayer:2012:ESI:2384616.2384666}).

Type systems are useful tools, particularly if they fit to the
programming language and the application scenario. If that is not the
case type systems can event get in the way. To ensure a type system
fits to a programming language, e.g.\ in the context of a \gls{dsl},
it makes sense to adapt and modify existing type systems or to create
new specialized type systems. Those specializations can lead to better
error messages, more expressive type systems and the detection of more
errors (cf.~\cite{Thiemann02programmabletype}
and~\cite{vandenBrand:2010:FST:1868281.1868293}). Currently there are
only some tools that support the creation and adaption of type systems
by generating type checkers from a type system specification
(\cite{Markstrum:2010:JDP:1667048.1667049}, \cite{ortin2014automatic},
\cite{gast2005generator}, \cite{tomb2005automatic} and
\cite{bergan2007typmix}), but they are not established. Such
generators accelerate the development of type checkers, make
development less error prone and fit well into the language
development workbench besides the long established lexer and parser
generators.

Type checker generators that use a high-level specification language
can also provide type theorists with a tool that allows them
experiment with new type systems without long development
delays. High-level specifications that are close to the formalism used
by type theorists have also the advantage of narrowing the gap between
type systems on paper and their implementation. Results shown for the
paper version can be adapted for the implementation (depending on the
correctness of the type checker generator). Such adaptions are not
possible with handcrafted implementations. High-level specification
languages are also suited to specify type systems from other areas,
like language-based information-flow
security~\cite{Sabelfeld:2006:LIS:2312191.2314769}.

A declarative specification can be translated into a first-order
formula representation, which can be used to conduct proofs of these
properties and thus raise the confidence regarding these
properties. As the proofs of propositions about type systems have to
change when the type system changes, it is desirable to make most of
these changes automatically. Automated theorem provers can try to
conduct those proves automatically. If the proof search fails one can
give hints for proof search, e.g.\ by providing induction
hypothesis. This establishes a direct connection between the specified
type system, the proofs and the generated type checker.

The translation of type system specifications into first-order formula
representations also enables to provably check for and remove
redundancies within the type system specification. As redundancies
have a potential to slow down the generated type checker, the removal
leads to different optimization strategies for the resulting type
checker.

The goal of this thesis is to narrow the gap between theory and
implementation of type systems. A declarative specification language
is developed, in which type systems are represented close to the
standard formalisms. Two things are generated from those
specifications. First, a representation of the type system as
first-order formulas, intended to support proving properties of the
type system. Second, an efficient type checker that exploits facts
proven by automated theorem provers for optimization.

\section{Contributions}
The main contributions of this thesis are:
\begin{enumerate}
\item A declarative specification language for static type systems,
  with support for natural deductive style typing rules and error
  messages. The specification language is implemented in \gls{sdf} and
  capable of using most \gls{sdf} definitions of programming
  languages.
\item A transformation of type system specifications into equivalent
  first-order formula representations in the \gls{tptp} format. The
  first-order formula representations are suitable for type checking
  using theorem provers, theorem proving and serve as a reference
  implementation for the type checker generator.
\item A generator that creates type checkers from type system
  specifications. This generator uses an intermediate representation
  for typing rules suitable for constraint generation and optimizes
  the intermediate representation using automated proofs, which check
  the applicability of optimizations.
\end{enumerate}

Figure~\ref{fig:interconnection} shows how these contributions are
connected. From a type system specification a first-order formula
representation and a type checker is generated and the first-order
formula representation is used in conjunction with an automated
theorem prover to check the applicability of optimizations for the
type checker.

\begin{figure}
  \begin{tikzpicture}[transform shape,->,>=stealth',shorten >=1pt,auto,align=center,node distance=1cm,thick,main node/.style={rectangle,fill=blue!20,draw,font=\small\bfseries}]
    \node[main node, style={ellipse}] (1) {Type System Specification};
    \node[main node] (2) [below left=2cm and .5cm of 1]  {First-order Formulas};
    \node[main node] (3) [below right=2cm and .5cm of 1] {Type Checker};

  \path[every node/.style={}]
  (1) edge node [outer sep=10pt,left] {Generation} (2)
  (1) edge node [outer sep=10pt,right] {Generation} (3);

  \path[every node/.style={}, dashed]
  (2) edge node [above] {used to validate \\ applicability of
    optimizations} (3);
  \end{tikzpicture}
\label{fig:interconnection}
\caption{Connection between theorem proving and type system optimization}
\end{figure}

The source code of the specification language, the first-order formula
transformation and the type checker generator including all type
system specifications is available as a Spoofax project at
\url{https://www.github.com/pSub/master-thesis}

\section{Structure}
\todo[inline]{Describe how the thesis is structured.}



%%% Local Variables:
%%% mode: latex
%%% TeX-master: "../report"
%%% End: