\chapter{Type System Specifications}
\section{SystemF}
\label{appendix:systemf}
\begin{lstlisting}[language=sltc]
module SystemF/Typesystem

imports SystemF/Freshness hiding (language)
        SystemF/Substitution hiding (language)

language specifications/SystemF/SystemF

contexts
TermBinding := ID{I} x Type{O}
TypeBinding := ID{I}

meta-variables  Term "~" { Type Exp }
                Ctx "$" { TermBinding TypeBinding }
                Id "%" { ID }
                Num "&" { Int }

judgments
TermBinding{I} "|" TypeBinding{I} "|-" Exp{I} ":" Type{O}.
Type{O} "= [" ID{I} "->" Type{I} "]" Type{I}.
ID{I} "!=" ID{I} is Neq.

rules

%x : ~T in $C1 @error %x "should have type" ~T "but has type" {}.
=============== T-Var
$C1 | $C2 |- %x : ~T

============== T-int
$C1 | $C2 |- &i : int

(%x : ~T1 ; $C1) | $C2 |- ~t2 : ~T2
@error ~t2 "should have type" ~T2 "but has type" {}.
==================================== T-Abs
$C1 | $C2 |- \ %x : ~T1 . ~t2 : ~T1 -> ~T2

$C1 | $C2 |- ~t1 : ~T11 -> ~T12
@error ~t1 "should have type" ~T11 "->" ~T12 "but has type" {}.
$C1 | $C2 |- ~t2 : ~T11
@error ~t2 "should have type" ~T11 "but has type" {}.
================================ T-App
$C1 | $C2 |- ~t1 ~t2 : ~T12

$C1 | (%x ; $C2) |- ~t2 : ~T2
@error ~t2 "should have type" ~T2 "but has type" {}.
%x fresh in $C2
======================================= T-Tabs
$C1 | $C2 |- \ %x . ~t2 : all %x . ~T2

~U = [ %x -> ~S ] ~T 
@error ~U "is not" ~T "where" %x "is replaced by" ~S.
$C1 | $C2 |- ~e : all %x . ~T
@error ~e "should have type all" %x "." ~T "but has type" {}.
============================== T-Tapp
$C1 | $C2 |- ~e [ ~S ] : ~U
\end{lstlisting}
\newpage
\begin{lstlisting}[language=sltc]
module SystemF/Freshness

language specifications/SystemF/SystemF

contexts TypeBinding := ID{I}

meta-variables  Ctx "$" { TypeBinding }
                Id "%" { ID }

judgments 
ID{I} "fresh in" TypeBinding{I}.
ID{I} "!=" ID{I} is Neq.

rules

==============
%x fresh in ()

%x != %y
%x fresh in $C
=====================
%x fresh in (%y ; $C)
\end{lstlisting}
\newpage
\begin{lstlisting}[language=sltc]
module SystemF/Substitution

language specifications/SystemF/SystemF

contexts

meta-variables  Term "~" { Type Exp }
                Id "%" { ID }

judgments
Type{O} "= [" ID{I} "->" Type{I} "]" Type{I}.
ID{I} "!=" ID{I} is Neq.

rules

===================== Subst-Eq
~S = [ %x -> ~S ] %x@1    @implicit %x " does not equal " %x@1.

%y != %x
==================== Subst-Neq
%y = [ %x -> ~S ] %y

~U = [ %x -> ~S ] ~T
========================================== Subst-All
(all %y . ~U) = [ %x -> ~S ] (all %y . ~T)

======================
int = [ %x -> ~S ] int

~U1 = [ %x -> ~S ] ~T1
~U2 = [ %x -> ~S ] ~T2
==================================== Subst-Arrow
~U1 -> ~U2 = [ %x -> ~S ] ~T1 -> ~T2
\end{lstlisting}

\section{Simply Typed Lambda Calculus with Subtying and Records}
\label{appendix:stlc-records}

\begin{lstlisting}[language=sltc]
module Typesystem

language specifications/Subtyping-algorithmic/SimplyTypedLambdaCalculus

contexts Context := ID{I} x Type{O}

meta-variables  Term "~" { Type Exp }
                Ctx "$" { Context }
                Id "%" { ID }
                R "$" {TRecordEntries RecordEntries}
                Num "&" { Int }

judgments
Context{I} "|-" Exp{I} ":" Type{O}.
TRecordEntries{I} "has" Exp{I} ":" Type{I}.
ID{I} "!=" ID{I} is Neq.

rules

/* Typing rules */

============== T-int
$C |- &i : int

%x : ~T in $C
============== T-var
$C |- %x : ~T

(%x : ~T1 ; $C) |- ~e : ~T2
================================= T-abs
$C |- \ %x : ~T1 . ~e : ~T1 -> ~T2

$C |- ~e1 : ~T -> ~S
$C |- ~e2 : ~T
========================= T-app
$C |- ~e1 ~e2 : ~S

$C |- ~e : { $R }
$R has %l : ~T
========================= T-proj
$C |- ~e . %l : ~T

============= T-empty
$C |- {} : {}

$C |- ~e : ~T
$C |- { $R } : { $S }
=====================================  T-record
$C |- { %l = ~e $R } : { %l : ~T $S }

%m != %l
$R has %l : ~T
====================== a
%m : ~T $R has %l : ~T 

====================== b
%l : ~T $R has %l : ~T
\end{lstlisting}
\newpage
\begin{lstlisting}[language=sltc]
module Subtyping

imports Typesystem hiding (language)

language specifications/Subtyping-algorithmic/SimplyTypedLambdaCalculus

contexts

meta-variables

judgments Type{I} "<:" Type{I}.

rules

$C |- ~e1 : ~T11 -> ~T12
$C |- ~e2 : ~T2
~T2 <: ~T11
========================= T-app
$C |- ~e1 ~e2 : ~T12

/* Definition of the subtyping relation */

======== S-refl
~S <: ~S

=========== S-ref
int <: int

~T1 <: ~S1
~S2 <: ~T2    
======================== S-arrow 
~S1 -> ~S2 <: ~T1 -> ~T2

/* Record Subtyping */

============ S-top
{ $R } <: {}

~T <: ~S
{ $R } <: { $U }
=============================== S-depth
{ %l : ~T $R } <: { %l : ~S $U }

======================================== S-width
{ %m : ~S %l : ~T $R } <: { %l : ~T $R }

======================================================== S-perm
{ %l1 : ~T1 %l2 : ~T2 $R } <: { %l2 : ~T2 %l1 : ~T1 $R }
\end{lstlisting}

\section{Information Flow Security Type System}
\label{sec:inform-flow-secur}

\begin{lstlisting}[language=sltc]
module Typesystem

language specifications/STWL/STWL

contexts Domain := ID{I} x Type{O}

meta-variables  Exp "~" { Exp Type }
                AExp "~1" { AExp }
                BExp "~2" { BExp }              
                Dom "$" { Domain }
                Id "%" { ID }
                Num "&" { Int }

judgments
// TODO: Make the domain implicit e.g. using the following notation
// [Domain{I}] "|-" AExp ":" Type{O}.
Domain{I} "|-" AExp{I} ":" Type{O}.
Domain{I} "|-" BExp{I} ":" Type{O}.
Domain{I} "|" Type{I} "|-" Exp{I}.

rules

================ num
$dom |- &n : low

%x : low in $dom
================ var
$dom |- %x : low

$dom |- ~1e1 : low
$dom |- ~1e2 : low
========================= opa
$dom |- ~1e1 + ~1e2 : low

================== true
$dom |- true : low

=================== false
$dom |- false : low

$dom |- ~2e : low
===================== not
$dom |- not ~2e : low

$dom |- ~1e1 : low
$dom |- ~1e2 : low
========================= opr
$dom |- ~1e1 < ~1e2 : low

$dom |- ~2e1 : low
$dom |- ~2e2 : low
=========================== opb
$dom |- ~2e1 and ~2e2 : low

================== higha
$dom |- ~1e : high

================== highb
$dom |- ~2e : high

================== skip
$dom | ~pc |- skip

$dom | high |- ~e
================= sub
$dom | low |- ~e

%x : high in $dom @error %x "should have type high".
======================== assgnh
$dom | ~pc |- %x := ~1e

$dom |- ~1e : low
======================= assgnl
$dom | low |- %x := ~1e

$dom | ~pc |- ~e
$dom | ~pc |- ~f
===================== seq
$dom | ~pc |- ~e ; ~f

$dom |- ~2b : ~pc
$dom | ~pc |- ~e
================================ while
$dom | ~pc |- while ~2b do ~e od

$dom |- ~2b : ~pc
$dom | ~pc |- ~e
$dom | ~pc |- ~f
==================================== ite
$dom | ~pc |- if ~2b then ~e else ~f

conjectures

===
(z : high ;()) | high |- if z < 0 then z := 0 else z := 1

// Order of domain makes a difference in proof search!
// For the next two conjectures the applied rules are given

// VerificationSuccess(["lookup base","var","assgnl","goal21","opa"])
===
(x : low ; y : low ; z : high ; ()) | low |- z := x + y

// VerificationSuccess(["lookup base","assgnh","goal23"])
===
(z : high ; y : low ; x : low ; ()) | low |- z := x + y

===
(x : low ; y : low ; ()) | low |- x := x + y

===
(x : low ; y : low ; ()) |- (x + 5) + y : high

/===
(x : low ; y : high ; ()) | low |- x := y

/===
(h : high; l : low ; ()) | high |- if h < 1 then l := 1 else skip
\end{lstlisting}

%%% Local Variables: 
%%% mode: latex
%%% TeX-master: "../report"
%%% End: 
