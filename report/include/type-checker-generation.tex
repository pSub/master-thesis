\chapter{Type Checker Generation}
\todo[inline]{Terminology: Type Checker vs. Type Inference}
\section{Goals}
This section briefly describes and motivates the design goals of the
type check generator.

The first goal is to produce fast type checkers, even if the type
system specification is not geared towards performance. Mainly this
means to try to deal with non-syntax directed typing rules without
backtracking. The motivation for this is, that non-syntax directed
typing rules are often more readable and changes have mainly local
effects.

The second goal is to design a modular type checker generator. A type
checker generator that can be easily adapted and that facilitates the
exchange of components. This modularity is desirable because it
increases the reusability and makes it possible to combine projects.

The third goal is to generate type checkers emit readable error
messages if a program is not well-typed. This is essential to make the
type checker usable in any way.
\section{Architecture}

\section{Implementation}
\subsection{Constraint Templates}
\subsection{Constraint Template Optimization}
\subsection{Constraint Generation}
\subsection{Constraint Solving}
\section{Performance}

%%% Local Variables: 
%%% mode: latex
%%% TeX-master: "../report"
%%% End: 
