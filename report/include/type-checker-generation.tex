\chapter{Type Checker Generation}
\section{Goals}
In this section we briefly describe and motivate the design and goals
of the type checker generator.

The first goal is to generate type checkers from high level type
system specifications that are not geared to type checking. Mainly
this means to try to deal with non-syntax directed typing rules
without backtracking. The motivation for this is that non-syntax
directed typing rules are often more readable and that changes have
mainly local effects. 

The second goal is to design a modular type checker generator,
particularly one that can be easily adapted and that facilitates the
exchange of components. This modularity is desirable because it
increases the reusability and makes it possible to combine previously
unrelated projects.

The third goal is to generate type checkers that emit readable error
messages if a program is not well-typed. This is essential to make the
generated type checker usable in production in any way.
\section{Architecture}
The type checker generator has two phases: Template generation and
template optimization. The template generation phase transforms the
type system specification with modifications into templates. A
template is a different representation for a typing rule that is
better suited for type checking. We introduce templates in detail in
Section~\ref{sec:constraint-templates}. The template optimization
phase checks which optimizations apply to the generated templates and
applies them. The optimizations aim to reduce the amount of
non-determinism in the type system and therefore reduce the amount of
backtracking in the type checker. The final product after those two
phases is a file that contains the optimized templates.

The template optimization phase is the key part of the type checker
generator. It is the link between the type checker generator on the
one hand and the formula generation and automated theorem provers on
the other hand. We generate from a single type system specification a
first-order formula representation suitable for theorem proving and a
template representation suitable for constraint generation. In the
optimization phase we combine them by using the automated theorem
prover to check if optimizations are applicable to the templates.

Instead of generating a type checker directly from the specification,
we implement a generic constraint-based type checker that takes
templates as input. This generic type checker has two phases:
Constraint generation and constraint solving. The constraint
generation phase takes the templates and the expression that shall be
type checked as input. We then build a derivation tree according to
the expression by pattern matching the conclusion of the
templates. The building and traversing of the derivation tree emits
constraints. In the constraint solving phase we then try to unify the
emitted constraints. If that succeeds it reports the result otherwise
it reports the errors that occurred during unification.

Figure~\ref{fig:phases} shows the phases and their
relationships. Nodes represent phases and arrows express that data
flows from one node to another. Labels on arrows describe the data
format.

The four phases correspond to modules or tools. Each phase has a
well-defined interface, therefore the implementation can be exchanged
freely. This facilitates the use of different constraint solvers,
constraint generators or template optimizers.

\begin{figure}
\begin{tikzpicture}[->,>=stealth',shorten >=1pt,auto,align=center,node distance=2cm,
  thick,main node/.style={rectangle,fill=blue!20,draw,font=\small\bfseries}]

  \node[main node] (1) {Template Generation};
  \node[main node] (2) [right=3cm of 1] {Template Optimization};
  \node[main node] (3) [below of=2] {Constraint Generation};
  \node[main node] (4) [below of=1] {Constraint Solving};
  \coordinate [below right of=3] (5);
  \coordinate [above of=1] (6);

  \path[every node/.style={font=\small}]
    (1) edge node [right, below] {Templates} (2)
    (2) edge node [right] {Extended Templates} (3)
    (5) edge node [right] {Program} (3)
    (3) edge node [right, below] {Constraints} (4)
    (6) edge node [right] {Specification} (1);
\end{tikzpicture}
\caption{Phases of the type checker generator}
\label{fig:phases}
\end{figure}

The following sections describe the implementation of the type checker
generator. Each section focuses on one of the phases.

\section{Template Generation}
\label{sec:constraint-templates}
The first phase translates type system specifications into
templates. This phase is not just a simple translation from a human
readable representation into a representation suitable for programs,
but also normalizes the resulting templates. Normalization comprises
the elimination of implicit equalities and resolves dependencies
between premises. All following phases assume normalized templates.
The normalization allows simplifications in the implementations of the
phases.

\subsection{Templates}
A template is an intermediate representation of a typing rule that is
more suitable for the constraint generation process and defined in
Definition~\ref{def:template}.

Templates serve as the input format for the constraint generation
phase of the type checker. Besides being better suited for constraint
generation templates also have the advantage that we can adapt the
system to other specification languages by translation into
templates. The structure of the templates has no hard requirements on
the specification language, although it has to be declarative. In
general the structure of the templates is similar to the structure of
the typing rules of the specification language.

\begin{figure}
\begin{definition}{Definition of the syntax of templates}
  The Stratego implementation appends to all constructors two
  underscores to avoid name collisions with target languages. This is
  necessary as the module system of Stratego is not strong enough to
  avoid those collisions. We omit these here for the sake of
  readability. The non-terminals Inputs, Outputs, Term and Error may
  contain arbitrary terms.
  \begin{grammar}
    <Template> ::= `Template' <Premises> <Conjecture>

    <Conjecture> ::= `Conjecture' <Judg> <Name> <Pattern> <Outputs>

    <Premises> ::= $\epsilon$ | <Premise> <Dependencies> <Premises>

    <Premise> ::= `Lookup' <Ctx> <Inputs> <Outputs> <Error>
    \alt `Judgment' <Judg> <Inputs> <Binding> <Outputs> <Error>
    \alt `Eq' <Term> <Term> <Error>
    \alt `Neq' <Term> <Term> <Error>

    <Dependencies> ::= $\epsilon$ | <Judg> <Outputs> <Dependencies>

    <Name> ::= `Some' <String> | `None'

    <Judg> ::= <Int>
  \end{grammar}
\label{def:template}
\end{definition}
\end{figure}


Before we are going to describe the structure of templates in depth,
we highlight the conceptual differences between templates and typing
rules. Templates take advantage of the input/output tags of typing
judgments and context definitions by splitting everything up into an
input and output part. This will become handy in the constraint
generation phase, where we can see immediately which are the patterns
to match against the expressions and which are the output positions
that we have to compute. As described in the introduction to this
chapter, we also normalize templates. For each implicit equality in
the conclusion and premises of typing rules we add an explicit
equality. Thus every variable occurs in each judgment only
once. Further we sort premises by dependencies between their
inputs/outputs. We describe and motivate the normalization process in
the remainder of the section. The template optimization and constraint
generation phase take advantage of these normalizations and therefore
expect their inputs to be normalized.

Note: We transform all meta-variables of the specification into the
variables (\code|Var(name)|) that we use for constraint generation as
we do not need the additional information of meta-variables anymore.

\subsubsection{Premises in Templates}
Premises in templates have multiple shapes: Judgments, context
lookups, equalities and inequalities. A premise always has a (possibly
empty) list of dependencies. This list contains the number and the
outputs of the judgment the premise depends on. Later on in this
section we define what it means for a premise to have dependencies,
first we take a look at the different kinds of premises.

\begin{description}
\item[Judgments] correspond to the user defined judgments of the
  specification. They consist of the judgment number
  \refcallout{judgment:number} which refers to the position in the
  declaration section of the specification, the positions of the
  judgment marked as inputs \refcallout{judgment:inputs}, context
  modifications \refcallout{judgment:bind},
  \refcallout{judgment:reset}, \refcallout{judgment:identity}, the
  output positions of the judgments \refcallout{judgment:outputs} and
  potentially error messages \refcallout{judgment:errors}.

  The main difference between context modifications in templates and
  typing rules is that all modifications are at one place and can
  always be evaluated in the same manner. A context modification can
  either be an addition to a context \refcallout{judgment:bind} which
  consists of a context identifier (this identifier refers to the
  position in the declaration section of the specification), a list of
  inputs and a list of outputs. It can also be a context identity
  \refcallout{judgment:identity} which corresponds to a context
  meta-variable in the specification and context identities do not
  modify the context and a reset operation \refcallout{judgment:reset}
  which resets the given context and corresponds to the empty
  context. Read to context modifications from right to left to obtain
  a valid context. A detail explanation of the semantics in our type
  checker will follow in Section~\ref{sec:constr-gener}.

\begin{example}{~}
\begin{lstlisting}[language=sltc]
Judgment(
        1 %*\callout{judgment:number}*)
      , [Var("X0")] %*\callout{judgment:inputs}*)
      , [ Binding(1, [Var("X1")], [Var("X2")]) %*\callout{judgment:bind}*)
        , Reset(1) %*\callout{judgment:reset}*)
        , Ctx(2) %*\callout{judgment:identity}*)
        ]
      , [Var("X3")] %*\callout{judgment:outputs}*)
      , Some([Error([Var("X0"), "has type", "{}"])]) %*\callout{judgment:errors}*)
      )
\end{lstlisting}
\label{ex:judgment}
\end{example}

In example~\ref{ex:judgment} we have judgment one that has only one
input position \code|Var("X0")| and one output position
\code|Var("X3")|. It leaves context two as it is, resets context one
and then adds the input/output pair \code|Var("X1")| and
\code|Var("X2")| to context one. In addition it is annotate with one
error message.

\item[Context lookups] consist of the context number
  \refcallout{lookup:ctx}, of the input \refcallout{lookup:inputs} and
  of the output \refcallout{lookup:outputs} positions of the context
  lookup from the specification and potentially of error messages
  \refcallout{lookup:errors}. We do not treat look ups as normal
  judgments as their semantics is not defined within the
  specification. Therefore we have to deal with them separately in the
  type checker.

\begin{example}{~}
\begin{lstlisting}[language=sltc]
Lookup(
        1 %*\callout{lookup:ctx}*)
      , [Var("X0")] %*\callout{lookup:inputs}*)
      , [Var("X1")] %*\callout{lookup:outputs}*)
      , None() %*\callout{lookup:errors}*)
      )
\end{lstlisting}
\label{ex:lookup}
\end{example}

  In example~\ref{ex:lookup} we look up \code|Var("X0")| from
  context one and bind the result to \code|Var("X1)"|. This context
  lookup has no error messages attached.

\item[(In)equalities] are predefined judgments and therefore treated
  separately. They have a judgment number, exactly two input
  positions, no output positions and potentially an error message. The
  judgment number is used to keep track of the non-terminals of the
  (in)equality.

\begin{example}
\begin{lstlisting}[language=sltc]
Neq(4, Var("X0"), Var("X1"), None())
\end{lstlisting}
\label{ex:neq}
\end{example}

In example~\ref{ex:neq} we test if \code|Var("X0")| and
\code|Var("X1")| are not equal and provide no error message.
\end{description}

In some cases it is relevant in which order we evaluate premises. As
we have not talked about evaluation yet, we assume premises are
evaluated like functions. We provide terms for input positions an
retrieve terms for the output positions. If we take a look at
Example~\ref{ex:premise-dependencies} we see that typing rule
\code|T-Tapp| from Appendix~\ref{appendix:systemf} has two
premises. When we compare the premises with the judgment definitions,
we see that \code|%x| occurs as an input of the first
premise and as within an output position of the second premise, but
not at all in the conclusion. If we want to evaluate the first premise
we have to find a term for
\code|%x|, but that term is only provided by
the output of the second premise. Therefore we have to evaluate the
second premise first.

\begin{example}{~}
\begin{lstlisting}[language=sltc]
judgments
TermBinding{I} "|" TypeBinding{I} "|-" Exp{I} ":" Type{O}.
Type{O} "= [" ID{I} "->" Type{I} "]" Type{I}.

rules
~U = [ %x -> ~S ] ~T
$C1 | $C2 |- ~e : all %x . ~T 
============================== T-Tapp
$C1 | $C2 |- ~e [ ~S ] : ~U
\end{lstlisting}
\label{ex:premise-dependencies}
\end{example}

We make those dependencies visible in the template language by
annotating each premise with a list of its dependencies. Those
dependencies contain the number of the premise on which the premise
depends and the output positions of that premise. The outputs are
redundant information and only added to make it easier to check which
information are proivded by that dependency.
The dependency for the first premise in
Example~\ref{ex:premise-dependencies} the would look like
\code|(1,[TAll(Var("X0"), Var("X1"))])|. In this example
\code|TAll(Var("X0"), Var("X1"))| is the only output provided by the
premise and \code|1| refers to the second premise, because we sort
premises of templates topological according to the dependencies, as we
will describe in the next section.

\subsubsection{Conclusions in Templates}
Conclusions contain the judgment identifier
\refcallout{conclusion:number}, the input positions of the judgment
\refcallout{conclusion:inputs}, the context modifications
\refcallout{conclusion:ctx} as well as the output positions of the
conclusion \refcallout{conclusion:outputs}. In contrast to the
conclusion in a typing rule from the specification, a typing rule in a
template has no error message. As it was only possible to annotate the
conclusion with error messages for implicit equalities those error
messages propagate into the premisses that are introduced to make the
implicit equalities explicit.

\begin{example}{~}
\begin{lstlisting}[language=sltc]
Conclusion(
      3 %*\callout{conclusion:number}*)
    , ( [Var("X177")] %*\callout{conclusion:inputs}*)
      , [ Binding(2, [Var("X178")], [])
        , Ctx(2)
        ] %*\callout{conclusion:ctx}*)
      )
    , [] %*\callout{conclusion:outputs}*)
    )
\end{lstlisting}
\label{ex:conclusion}
\end{example}

Example~\ref{ex:conclusion} shows a conclusion that has judgment
number three and only one input position. The context pattern specifies
that there has to be at least one element in the second context. In
this example judgment three has no outputs.

A template consists of a name, a list of premisses with dependencies
and a conclusion. Figure~\ref{fig:template-example} shows how a
complete template looks like for the variable typing rule in the
simply typed lambda calculus. On the left side of
Figure~\ref{fig:template-example} the typing rule from the
specification is shown and on the right side its representation as a
template.

\subsection{Generation}
The Stratego rule \code|to-template| does the main part of the
template generation. It takes a rule from a specification and
transforms that rule into a template. The strategy \code|to-templates|
does that iteratively for every rule in a specification. The first
step in the conversion is the elimination of implicit equalities in
the premises and the conclusion.

For each implicit equality we create an explicit equality, by
collecting all variables that occur more than once in an input
position of a premise or conclusion. After that we create fresh names
for the collect variables and create premises that state the equality
of the fresh meta-variables. Implicit equalities in premises are not
transformed into explicit equalities if the meta-variables also occur
in the conclusion. These equalities are either ensured by explicit
equalities from the conclusion or if there are no implicit equalities
in the conclusion for this meta-variable, by the fact that this
variable occurs only once as a source and thus, cannot introduce
different values. If a meta-variable occurs more than twice we define
the equalities for the new variables transitively.

Example~\ref{ex:implicit-equalities} shows this for a typing rule of
the judgment

\begin{lstlisting}[language=sltc]
Type{O} "= [" ID{I} "->" Type{I} "]" Type{I}
\end{lstlisting}

which models type substitution in the SystemF specification in
Appendix~\ref{appendix:systemf}. On the left side of
Example~\ref{ex:implicit-equalities} you see the typing rule with an
implicit equality and on the right hand side the transformed version
without the implicit equality.

\begin{example}{~}
\newline
  \begin{minipage}[b]{.45\linewidth}
    \begin{lstlisting}[language=sltc]
=====================
~S = [ %x -> ~S ] %x
\end{lstlisting}
  \end{minipage}
  \begin{minipage}[b]{.45\linewidth}
    \begin{lstlisting}[language=sltc]
%y = %z
=====================
~S = [ %y -> ~S ] %z
\end{lstlisting}
  \end{minipage}
\label{ex:implicit-equalities}
\end{example}

If there are error messages for implicit equalities, we associate
those error messages with the corresponding implicit equalities. Then
we replace the variables in the error message by the fresh
meta-variables of the corresponding explicit equality and add the
error message as a normal error to the explicit equality.

In order to treat the introduced explicit equalities like the
equalities introduced in the type system specification, we have to
introduce corresponding judgments. This is important as we rely in the
template optimization phase on the information about the non-terminals
in the judgments.

To introduce judgments for equalities that are generated from implicit
equalities, we have to determine the non-terminals of the
equality. We infer the non-terminals by analyzing the surrounding
terms and by exploiting the scope of the meta-variable.

In case the meta-variable is directly in an input or output position
of a judgment, context binding or context lookup, we infer the
non-terminal from the judgment or context definition. Otherwise we
fetch all productions of the target language that could possibly have
produced the surrounding term and extract the non-terminal from the
corresponding position. If there is more than one production that
could have created that term, we try to disambiguate by the scope of
the meta-variable. In case there is still more than one production
with different non-terminals for this meta-variable, we raise an
exception and this equality has to be made explicit in the
specification.

After this step, there are no implicit equalities left, i.e.\ all
meta-variables within a judgment are different. As a next step we
analyze the premises of the typing rule for dependencies. One premise
depends on the other, if it uses meta-variables in input positions
that occur in an output position of another premise and not as an
input position in the conclusion. As we need to supply the input
positions with concrete terms to check if a premise holds, we can only
evaluate a dependency if we first evaluate all its dependencies. In
the generation process we do two things: We sort the premises
topologically according to their dependencies and annotate them with
the output positions of the dependencies, as described in the previous
paragraph. The implementation of this ordering is almost
standard. First we collect all premises and their dependencies in a
graph. Due to the nature of the dependency it is more natural to
create edges from a premise to the premises it depends on. As it is
easy to check if a premise has an input position that does not occur
in the conclusion but as the output of another premise. Therefore the
resulting dependency graph is a transposed dependency graph. We sort
this graph topologically with the algorithm by
Kahn~\cite{Kahn:1962:TSL:368996.369025}. The result is the reversed
order, as we used the transposed dependency graph. Therefore we
reverse the result of the topological sort. If we encounter cycles in
the dependency graph we report them and abort the template generation.

After we have resolved implicit equalities and dependencies we begin
to generate templates. This process is, except for technical
subtleties, a straight forward rewriting of the rules from the
specification.

\begin{figure}
  \centering
  \begin{minipage}{.35\linewidth}
\begin{lstlisting}[language=sltc]
%x : ~T in $C
============== T-var
$C |- %x : ~T
\end{lstlisting}
  \end{minipage}
  \begin{minipage}{.55\linewidth}
\begin{lstlisting}[language=sltc]
Template(
    Some("T-var")
  , [ ( Lookup(
          1
        , [Var("X52")]
        , [Var("X53")]
        , None()
        )
      , []
      )
    ]
  , Conclusion(
      1
    , ([Var(Var("X52"))], [Ctx(1)])
    , [Var("X53")]
    )
  )
\end{lstlisting}
  \end{minipage}
  \caption{Typing rule and template of T-Var}
  \label{fig:template-example}
\end{figure}
\section{Constraint Template Optimization}
\label{sec:constr-templ-optim}
In this section we describe and motivate the optimization strategies
we have used. The template optimization phase can potentially do
arbitrary modifications. We focus here on optimization strategies that
reduce ambiguities between the templates.

Informally two templates are ambiguous if it is not known a priori
which (if any) of the templates we have to use to extend the
derivation tree to create a successful derivation. Therefore we have
to use the rules according to a heuristic and to backtrack if the
decision did not work out. The goal of the optimization strategies
described in this section, is to rule out some of those ambiguities
before we even start to create derivations.

Before introducing the optimization strategies, we define more
formally what an ambiguity is.

\begin{definition}
  A template is \textit{when-ambiguous} if there is another template
  such that there is at least one term that matches the conclusion of
  both templates.
\label{def:when-ambiguous}
\end{definition}

Definition~\ref{def:when-ambiguous} describes the general case of an
ambiguity. There is an ambiguity if there is a template that has an
syntactic overlap with another template. We call those ambiguities
when-ambiguity, because it is not clear when we have to apply an
ambiguous template. 

\begin{definition}
  A template is \textit{which-ambiguous} if there is another template
  such that the set of terms matching the conclusion of both templates
  is equal.
\label{def:which-ambiguity}
\end{definition}

Definition~\ref{def:which-ambiguity} aims at more specialized
ambiguities. Templates are when-ambiguous if they match exactly the
same terms. As there are two (or more) templates that apply to exactly
the same terms, it is not the question \textit{when} we have to apply
a template, but only \textit{which} one we have to apply. The
distinction of those two kinds of ambiguities will help to implement
optimization strategies.

Before describing the optimization strategies we implemented, we
extend the template language to make ambiguities explicit.

\begin{definition}
    The following grammar defines the extended template language.
    \begin{grammar}
    <Template> ::= $\dots$ | `Fork' <Templates>

    <Templates> ::= <Template> | <Template> <Templates>
    \end{grammar}
\end{definition}

The new constructor \code|Fork| has a list that contains at least one
template. The idea is that templates contained in \code|Fork|s are all
which-ambiguous to each other. We later use those groups of ambiguous
templates to decide at which point in the derivation we have to
decide which template we apply.

Now we look now into the optimization of which-ambiguities. First we
identify all which-ambiguities and create \code|Fork|s of them. We
implement using a strategy that groups arbitrary lists according to a
comparison strategy.\todo{context} Two templates are which-ambiguous
if they have the same judgment number and if the term pattern and
context pattern of the conclusion are equal modulo variables. Of the
resulting template groups we wrap all groups with more than one
element into a \code|Fork|.

We optimize forks containing which-ambiguities by checking whether a
template of a fork is subsumed by an other template of the same
fork. Template $t_1$ subsumes template $t_2$ if all premises of $t_2$
imply at least one premise of $t_1$. This is sufficient, because the
conclusion of all templates within a fork matches the same terms.

In Example~\ref{ex:which-ambiguity} we have the depth and width
subtyping rules for records. Those two rules are which-ambiguous to
each other. Their conclusions contain the same judgment. They have no
contexts and exactly the same term pattern.

\begin{example}{~}
\begin{lstlisting}[language=sltc]
======== S-Refl
~T <: ~T

~T <: ~S
{ $R } <: { $U }
=============================== S-depth
{ %l : ~T $R } <: { %l : ~S $U }

~T = ~S
{ $R } <: { $U }
=============================== S-width
{ %l : ~T $R } <: { %l : ~S $U}
\end{lstlisting}
\label{ex:which-ambiguity}
\end{example}

We know that the subtyping relation \code|<:| is reflexive by the rule
\code|S-Refl|. Now we show that all premises of \code|S-width| imply
at least one premise of \code|S-depth|. \code|~T = ~S| implies
\code|~T <: ~S| because the subtyping relation is reflexive and
\code|{ $R } <: { $U }| occurs verbatim as a premise of
\code|S-depth|. Therefore the depth subtyping rule subsumes the width
subtyping rule and we can remove the width subtyping rule.

We implement this checking whether there exists for each template $t$
within a fork a template $t'$ such that

\begin{align}
  (p_1 \implies (q_1 \lor \dots \lor q_m)) \land \dots \land (p_n \implies
  (q_1 \lor \dots \lor q_m))
\label{formula:which-ambiguity}
\end{align}

holds. In Formula~\ref{formula:which-ambiguity} $p_i$ are the
premises of $t$ and $q_i$ are the premises of $t'$. We verify the
conjuncts in Formula~\ref{formula:which-ambiguity} separately with the
vampire theorem prover. If vampire succeeds for all conjuncts we
remove template $t$. 

At the end of the optimization phase we sort the templates in the
remaining forks. Forks that have a specialized conclusion are in front
of forks with a general conclusion. This ordering has the advantage
that we can evaluate the templates in a fork from left to right, so
that no general template will catch all terms. More formally, we sort
the templates within a fork by $<$, which is defined as

\begin{align}
  t_1 < t_2 \iff m(t_1) \subseteq m(t_2)
\end{align}

where $m(t)$ computes the set of terms that match the conclusion of
$t$. We implement the comparison as a Stratego strategy and use the
quick sort implementation of the Stratego standard library. The
comparison strategy does the following: It checks whether the judgment
number is equal, which is always satisfied as we compare templates
within a fork. Then it checks whether the constructors of the term and
context pattern of the left operand match the constructors of the term
and context pattern of the right operand modulo variable names.

\section{Constraint Generation}
\label{sec:constr-gener}
Constraint generation is the first phase of the generated type
checker. The inputs of this phase are constraint templates and the
expressions that should be type checked. The only input method for
expression we currently support is via the conjecture section of the
specification. However it should be straight forward to extend this
phase with support for e.g.\ \code|stdin|.

The constraint generation phase is structured as follows. First we
initialize contexts and then generate constraints according to the
program's structure and the templates. After generating the constraints
we annotate whether the expression should have been well-typed or not.

Lets first look at the implementation of contexts. We store all
contexts in a hash table, with the context ID as key. The contexts
itself are hash tables extended with a list that stores the insertion
order. Hash tables are a convenient data structure in the Stratego
standard library to store (multiple) data points at a key. They
reflect the intuition of the context declarations as cross products of
the terms corresponding to non-terminals. It is important to track the
insertion order, because some type systems (e.g.\ ordered type
systems) might rely on the ordering within contexts.

For the context representation we decided to use hashtables instead of
normal key-value lists to reduce the lookup time in realistic programs
which declare identifiers much less then they refer to them. However
we do not have experimental evidence if this actually speeds up type
checking, because we do not have a key-value list implementation of
contexts.

Implementing contexts using extended hash tables forced us to separate
context operations from normal inputs. On the one hand this creates
more exceptional cases (e.g.\ the matching complex matching algorithms
described in Section~\ref{sec:constr-templ-optim}) and on the other
hand it leads to a modular context implementation with the potential
to exchange the implementation easily.

Before we can explain how we generate constraints, we have to define
the constraint language.

\begin{definition}{Constraint Language}
  \begin{grammar}
    <Constraint> ::= `CFail' <Error>
    \alt `CEq' <Term> <Term> <Error>
    \alt `CNeq' <Term> <Term> <Error>
  \end{grammar}
\end{definition}

We annotate all constraints with error messages. These error messages
are either generated during the constraint generation or passed
through from the specification. The constraint solver shows the error
messages if it cannot solve a constraint. There are three types of
constraints. The constraint \code|CFail| corresponds to false and
cannot be solved. Further \code|CEq| and \code|CNeq| are equality
respectively inequality constraints.

As we read the program expressions from the conjecture section of the
specification, we already have the initial contexts, the judgment
number and we can tell which are input and output positions. If we
would only read the program expression (e.g.\ from \code|stdin|) we
would need to pass information about the initial contexts, the
judgment number and the output positions separately. Every conjecture
has a fresh context store, initialized with the initial contexts
provided in the conjecture.

We have divided the main logic of constraint generation into three
Stratego strategies: \code|generate| selects the template matching the
current term, \code|generate'| applies the current term to the
conclusion of the selected template and \code|execute| evaluates the
premises of the template. All three strategies return a tuple of the
computed outputs and the generated constraints. We wrap the computed
outputs into \code|Option| to model the absence of outputs in cases of
errors.

After initializing the contexts we call the strategy \code|generate|
with the judgment number and the inputs of the conjecture. We also
pass the expected outputs and set the parameter \code|init| to
\code|true|. This parameter is \code|false| for all other cases except
the initial call to \code|generate| and used to generate an equality
constraint for the expected and actual outputs.

Now \code|generate| calls \code|find-match| to find a template whose
conclusion matches the inputs, judgment number and the current state
of the contexts. First \code|find-match| filters all templates with
the correct judgment number and tests only those templates for
matches. \todo{Restructure sentence, First is misleading} In case of
\code|Fork| only those templates are retained that match. For all
other templates \code|find-match| checks whether the constructors of
the term pattern in the conclusion and of the input term match. They
can have variables for whole levels and match if they have the same
constructors on all levels. In addition the context pattern of the
conclusion has to reflect the current state of the contexts. The
context pattern reflects the state of the contexts if it can be
applied from left to right to the context without failures.

\code|find-match| safely takes the first template that matches,
because in Section~\ref{sec:constr-templ-optim} we have ensured that
there is only one template that matches. If no template matches, we
return \code|None| for the outputs and the singelton list containing
the constraint that always fails, with an appropriate error
message. If \code|find-match| succeeds to find a template it calls
\code|generate'| with that template.

The rule \code|generate'| now evaluates the selected template
according to the given input. In \code|generate'| we use patterns to
access specific elements of terms. The corresponding element to the
input term is the term pattern of the templates conclusion. We update
the terms and patterns if we want to add another term and pattern to
the scope. The patterns and terms are later used to instantiate
variables in premises or error messages.

Unless the selected rule is no fork, the evaluation works as
follows. In correspondence to the bindings in the context pattern of
the conclusion, we pop all terms from the contexts and bring them into
scope. Also if a variable occurs in the inputs of a premise but
neither in the inputs of the conclusion nor in the outputs of other
premises but int the outputs of the conclusion, then we bring it into
scope with the expected outputs. \todo{restructure sentence} This
allows to deal with typing rules that have free variables in their
premises as for instance in the subsumption rule for subtyping.

\begin{example}{~}
\begin{lstlisting}[language=sltc]
$C |- ~e : ~S
~S <: ~T
============= T-Sub
$C |- ~e : ~T
\end{lstlisting}
\label{ex:subsumption}
\end{example}

Example~\ref{ex:subsumption} shows the subsumption rule from
\todo{Cite}. In the rule \code|~T| occurs free in the second
premise. Binding it to the expected output of \code|~T| allows to
still evaluate this premise.

Now we have to replace all variables introduced by the selected
template with fresh variables, otherwise applying a template twice
would lead to collisions. All variables introduced in the previous
phases have a prefix different from \code|Y|. Therefore we replace all
variables that do not have the prefix \code|Y| with a fresh variable
with the prefix \code|Y|. Additionally this allows to verify visually
whether all variables are fresh.

Now we evaluate the premises of the template. A key element in the
evaluation is the instantiation of the variables in the premise
patterns according to the conclusion pattern and input term, because
we need input terms to recursively call \code|generate| on the
premises or to create solvable (in-)equality
constraints. \todo{split sentence up}

We initialize variables in inputs of a judgment using a term and a
pattern which describes the structure of the term. We search a
variable in the pattern to instantiate it, by walking through the
abstract syntax tree of the pattern and record the path to the
variable. Every variable is distinct as we resolved every implicit
equality in the templates and have introduced no new implicit
equality. Therefore we can take the first and only occurrence of the
variable that matches. If we find a path to that variable, we use it
to retrieve terms from the corresponding term, by walking along the
path and fetching the node at its end. We ensure that the pattern and
term always have the same structure, otherwise we would retrieve false
positives.

As described in Section~\ref{sec:constraint-templates} the premises
were topologically sorted we can evaluate them in the given
order. However, to ensure that the terms of all dependencies are
available during evaluation, we have to accumulate the outputs of the
evaluated premises and add those to the patterns and terms used for
the instantiation.

The strategy \code|execute| does the evaluation of the premises. We
call \code|execute| for each premise with a copy of the contexts to
ensure that one premise cannot pollute the contexts of another
premise. We have implemented a safe version of \code|hashtable-copy|
to ensure that after the evaluation of the premise all copies are
destroyed.

We now describe what \code|execute| does for the different types of
premises.

\begin{description}
\item[Lookup:] The inputs of the lookup are instantiated, then the
  resulting term is looked up in the corresponding context. If the
  lookup succeeds the result is returned as an output, together with a
  constraint set that contains equality constraint between the output
  pattern and the looked up output. Those constraints are annotated
  with the corresponding error message, which is instantiated and in
  which the hole (\code|{}|) is replaced by the output variables. If
  the lookup fails \code|None| is return as output together with the
  \code|CFail| constraint and an appropriate error message.
\item[(In)equality:] Equalities and inequalities are initialized and
  returned as constraints, together with the output \code|None| as
  they have no output positions.
\item[Judgment:] If the premise is a judgment it is first
  instantiated. Then we call \code|generate| on the resulting input
  term with a version of the store that has been updated according to
  the context pattern of the judgment. In addition we instantiate the
  expected outputs of the judgment to make the binding of free
  variables in premises more precise. After the call to
  \code|generate| returns we generate equality constraints for the
  bindings in the context pattern of the premise and generate equality
  constraints between the output pattern and the computed output, with
  corresponding instantiated error messages. \todo{split up sentence}
  The hole (\code|{}|) in the error messages is here also replaced by
  the output variables.
\end{description}

In the case \code|generate'| encounters a fork it evaluates all
templates contained in the fork by calling \code|generate'|. It tries
to solve the resulting constraint set and returns the outputs of the
first evaluation that succeeds. If no templates can be evaluated to a
solvable constraint set we return the constraint that always fails
with an appropriate error message. Because of the ordering of forks in
Section~\ref{sec:constr-templ-optim} this procedure ensures that rules
that potentially that produce solvable constraint sets but do not make
real progress are deferred to the end.

\section{Constraint Solving}
\label{sec:constraint-solving}
Constraint solving is the last phase of the type checker. The
constraint language is simple because it only consists of three
constructs, the algorithm to solve a constraint set is simple as well.

We use unification to solve the constraint sets. To be even more
precise we use a variant of Robinson unificiation
\todo{Citation}. During the unification we compute a \gls{mgu} to
instantiate the variables in the outputs and (in case of ill-typed
programs) in the error messages that we collect every time a
constraint cannot be solved.

During unification we ensure that a certain kind of malformed
constraint set does not lead to infinite loops in the unification. If
there are only inequalities left that contain at least one variable,
we abort unification. As inequality is not transitive, we can never
solve those constraint sets.
\section{Performance}
\label{sec:performance}

%%% Local Variables: 
%%% mode: latex
%%% TeX-master: "../report"
%%% End: 
