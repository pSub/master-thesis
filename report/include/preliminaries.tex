\chapter{Preliminaries}
\section{Tools}
This section introduces the tools used in this thesis. We argue what
makes the tools suitable for achieving the goals of the thesis and
give a short overview of the alternatives.

\subsection{SDF}
\gls{sdf}~\cite{Heering:1989:SDF:71605.71607} is a formalism to define
the syntax of formal languages in the tradition of
\gls{bnf}. \gls{sdf} allows to define lexical and context-free
grammars and is, in contrast to \gls{bnf}, modular. This modularity
allows to compose \gls{sdf} syntax definitions. There are also parser
generators for \gls{sdf} (see~\cite{Rekers92parsergeneration}). In
contrast to formalisms for traditional parser generators like
Yacc~\cite{Johnson75yacc:yet}, \gls{sdf} specifications are purely
declarative, which increases the reusability of the specifications.

The modularity of \gls{sdf} is a consequence of the fact that
\gls{sdf} is \textit{generalized LR
  parsable}~\cite{Rekers92parsergeneration}. Generalized parsing means
that parsing might be indeterminate, i.e.\ the parser produces all
possibilities for syntactic ambiguities. This might not seem to be an
advantage, but generalized parsing has some nice implications. First,
the class of supported grammars is bigger and therefore enforces less
restrictions on the programmer and might allow a natural definition of
languages. Second, it enables the modularity of \gls{sdf}, because the
composition of generalized LR grammars is a generalized LR
grammar. This does not hold in general for LR grammars.

An other advantage is that \gls{sdf} is also scannerless
parsable~\cite{Brand02disambiguationfilters}, that means that
tokenization and parsing can be done in a single step.

\gls{sdf} also has support to specify layout sensitive
languages~\cite{conf/sle/ErdwegRKO12} like Python or Haskell. It
enforces context insensitive layout constraints at parse time and
context sensitive constraints at disambiguation time, i.e.\ all
ambiguities that violate layout constraints are removed.

\begin{figure}
\begin{example}{~}
\begin{lstlisting}[language=sdf]
module Bool
exports
  context-free start-symbols Bool
  sorts Bool
  context-free syntax
    "true"  -> Bool
    "false" -> Bool
    "~" Bool -> Bool {cons("Not")}
    Bool "&" Bool -> Bool {cons("And"),
                           layout("2.first.col < 2.left.col")}
\end{lstlisting}
\label{ex:sdf-grammar}
\end{example}
\end{figure}

Example~\ref{ex:sdf-grammar} shows a grammar that allows to write
conjunctions of \code|true| and \code|false|. Conjunctions are
represented in the \gls{ast} with the constructor \code|And| and the
layout constraint enforces, that if conjunctions span over multiple
lines, the right-hand side of \code|&| needs to be indented.

\gls{sdf} is used in this thesis to define the syntax of the
specification language for type systems and for the target
languages. Features like layout constraints make \gls{sdf} a good tool
for defining complex syntax for natural deduction rules. The
modularity and composability make it possible to define the syntax of
the specification language independent of the target language.

\subsection{Stratego/XT}
Stratego~\cite{Visser01} is a framework for the development of
transformation systems. It consists of the transformation language
Stratego and a set of tools (XT) for tasks like parsing and pretty
printing. The paradigm of Stratego is to use user-definable strategies
for rewriting. Stratego distinguishes the following abstraction
levels.

\begin{description}
\item[Transformation rules] are basic rewrite rules on the structure
  of the \gls{ast}.
\item[Transformation strategies] are the glue between the
  transformation rules. They combine rules, define the order of
  application and the traversal order of the \gls{ast}. These
  strategies can be defined generically. \textit{scoped dynamic
    rewrite rules}~\cite{Visser01scopeddynamic} pass context
  information during the traversal, because normal rewrite rules are
  context-free.
\item[Transformation tools] allow to compile transformation strategies
  into a stand-alone program. The interface between such programs is
  the ATerm format for \glspl{ast}.
\item[Transformation systems] describe a set of programs created by
  the transformation tools. A transformation system for a
  source-to-source transformation usually includes a parser and pretty
  printer.
\end{description}

Example~\ref{ex:stratego-module} shows a Stratego module that declares
the constructors from Example~\ref{ex:sdf-grammar}. It has one rule
\code|Eval| that reduces a term one step. The strategy \code|eval|
applies rule \code|Eval| repeatedly from bottom to top.

\begin{figure}
\begin{example}{~}
\begin{lstlisting}[language=stratego]
module Bool
signature
  sorts Bool
  constructors
    Not   : Bool -> Bool 
    And   : Bool * Bool -> Bool
rules
  Eval : Not(True)      -> False
  Eval : Not(False)     -> True
  Eval : And(True, x)   -> x
  Eval : And(x, True)   -> x
  Eval : And(False, x)  -> False
  Eval : And(x, False)  -> False
strategies
  eval = bottomup(repeat(Eval))
\end{lstlisting}
\label{ex:stratego-module}
\end{example}
\end{figure}

Stratego interacts well with \gls{sdf} and has facilities to integrate
into tool-chains. In addition it allows to write abstract and generic
transformations, which makes it a good choice for the transformation
system.

\subsection{Spoofax}
Spoofax~\cite{KatsV10} started with the goal to provide an \gls{ide}
for \gls{sdf} and Stratego. It was then developed into a language
workbench for Eclipse that allows language development with editor
support for both, the languages used for development and the developed
language. Spoofax allows smooth switching between both editor
services. The editor for the developed language can be deployed
standalone. Those editor services can provide syntactic and semantic
analysis based on live parses, with error recovery and origin
tracking. Those facilities are implemented language pragmatic, which
allows developers to focus on language specific parts.

We use Spoofax in this thesis to provide editor support for the
specification language and as glue between \gls{sdf} and Stratego in
the development of the specification language, the formula generator
and the type checker generator.

\subsection{Vampire}
We use the automatic theorem prover Vampire~\cite{VoronkovVampire} to
prove classical first-order logic propositions about type system
specifications. Vampire is able to parse formulas in the \gls{tptp}
format and provides detailed output for the conducted proofs. We use
this output to extract information about the applied axioms, which are
in our case typing rules.

\subsection{Alternatives}
We chose Spoofax and its components, because its feature set fit well
to our goals and there was some previous work related work based on
it. However there are alternative \textit{language workbenches} that
could have been a good fit as well. For example
Rascal~\cite{klint2009rascal} a meta-programming language that has
among others support for context-free grammars, algebraic data-types,
relations, relational calculus operators, advanced patterns matching,
generic type-safe traversal, comprehensions, and string templates for
code generation.~\cite{workbenches} gives an overview of the state of
the art in language workbenches.

\section{Type Systems}
\textit{Type theory} started as an attempt by Gottlob Frege to solve
Russel's paradox, that shows that \naive{} set theory is
inconsistent. Frege argued that a predicate requires an object as
argument and cannot have itself as an argument, as it is the
subject. So the initial motivation for type theory was to avoid
paradoxes and contradictions in logics and rewrite systems. The term
\textit{type system} refers to type theories whose logics rewrite
systems are programming languages. Type systems address the problem to
ensure that programs have meaning, whereas in type theory the problem
is the ensure the consistency of a logic.

What does it mean to ``ensure that programs have meaning''? First, it
means that one wants to filter the useful programs. It is not useful
to have a syntactically valid program that has no semantics. A type
system assigns types to the expressions of the programming language. A
\textit{type checker} can then verify, whether the types of the
programs expressions match according to predefined typing rules. If
the type checker succeeds the program is \textit{well-typed} and has
meaning, i.e.\ the program will not exit due to undefined
semantics. Depending on the expressiveness of the type system also the
correctness of the computation that the program performs can be shown.

\section{First-Order Logic}

%%% Local Variables: 
%%% mode: latex
%%% TeX-master: "../report"
%%% End: 
