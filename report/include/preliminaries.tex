\chapter{Preliminaries}
\label{cha:preliminaries}
\section{Tools}
This section introduces the tools used in this thesis. We argue what
makes the tools suitable for achieving the goals of the thesis and
give a short overview of the alternatives.

\subsection{SDF}
\glsreset{sdf}
\gls{sdf}~\cite{Heering:1989:SDF:71605.71607} is a formalism to define
the syntax of formal languages in the tradition of the
\gls{bnf}. \gls{sdf} allows to define lexical and context-free
grammars and is, in contrast to \gls{bnf}, modular. This modularity
allows to compose \gls{sdf} syntax definitions. There are also parser
generators for \gls{sdf} (see~\cite{Rekers92parsergeneration}). In
contrast to formalisms for traditional parser generators like
Yacc~\cite{Johnson75yacc:yet}, \gls{sdf} specifications are purely
declarative, which increases the reusability of the specifications.

The modularity of \gls{sdf} is a consequence of the fact that
\gls{sdf} is \textit{generalized LR
  parsable}~\cite{Rekers92parsergeneration}. Generalized parsable
means that parsing might be indeterminate, i.e.\ the parser produces
all possibilities for syntactic ambiguities. This might not seem to be
an advantage as we have to deal with ambiguities now, but generalized
parsing has some nice implications. First, the class of supported
grammars is larger and therefore enforces less restrictions on the
programmer and might allow more natural definitions of formal
languages. Second, it enables the modularity of \gls{sdf}, because the
composition of generalized LR grammars is a generalized LR
grammar. This does not hold in general for LR grammars.

Another property of \gls{sdf} is that it is scannerless
parsable~\cite{Brand02disambiguationfilters}, i.e.\ tokenization and
parsing can be done in a single step. The advantages are that only one
meta-language is needed and non-regular lexical structure are handled
easily.

\gls{sdf} supports specification of layout sensitive
languages~\cite{conf/sle/ErdwegRKO12} like Python or Haskell. It
enforces context insensitive layout constraints at parse time and
context sensitive constraints at disambiguation time, i.e.\ all
ambiguities that violate layout constraints are removed.

\begin{figure}
\begin{example}{~}
\begin{lstlisting}[language=sdf]
module Bool
exports
  context-free start-symbols Bool
  sorts Bool
  context-free syntax
    "true"  -> Bool
    "false" -> Bool
    "~" Bool -> Bool {cons("Not")}
    Bool "&" Bool -> Bool {cons("And"),
                           layout("2.first.col < 2.left.col")}
\end{lstlisting}
\label{ex:sdf-grammar}
\end{example}
\end{figure}

Example~\ref{ex:sdf-grammar} shows a grammar that allows to write
boolean expressions. Conjunctions are represented in the \gls{ast}
with the constructor \code|And|, written as an infix \code|&|. The
layout constraint enforces, that if conjunctions span over multiple
lines, the right-hand side of \code|&| needs to be indented. Negation
is represented in the \gls{ast} with the constructor \code|Not| and
written as a prefix \code|~|. True and false are written as
\code|true| and \code|false| respectively. In this example all
expressions belong to the same syntactic sort \code|Bool|. Sorts
represent non-terminals in \gls{sdf}. The start symbol is \code|Bool|
as we only have a single sort.

\gls{sdf} is used in this thesis to define the syntax of the type
system specification language. Programming languages used in
specifications are defined in \gls{sdf}, too. Features like layout
constraints make \gls{sdf} a good tool for defining complex syntax for
natural deduction rules. Further modularity and composability make it
possible to define the syntax of the specification language
independent of the target language.

\subsection{Stratego/XT}
Stratego~\cite{Visser01} is a framework for the development of
transformation systems. It consists of the transformation language
Stratego and a set of tools (called XT) for tasks like parsing and
pretty printing. The approach of Stratego is to use user-definable
strategies for rewriting. Stratego distinguishes the following
abstraction levels.

\begin{description}
\item[Transformation rules] are basic rewrite rules on the structure
  of the \gls{ast}.
\item[Transformation strategies] are the glue between the
  transformation rules. They combine rules, define the order of
  application and the traversal order of the \gls{ast}. These
  strategies can be defined generically. \textit{scoped dynamic
    rewrite rules}~\cite{Visser01scopeddynamic} allow to pass context
  information during the traversal, because transformation rules and
  strategies are context-free.
\item[Transformation tools] allow to compile transformation strategies
  into a stand-alone program. The interface between such programs is
  the ATerm format~\cite{VandenBrand:2000:EAT:343460.343468} for
  \glspl{ast}.
\item[Transformation systems] describe a set of programs created by
  the transformation tools. A transformation system for a
  source-to-source transformation usually includes a parser and pretty
  printer.
\end{description}

Example~\ref{ex:stratego-module} shows a Stratego module that declares
the constructors from Example~\ref{ex:sdf-grammar}. It has a rule
\code|Eval| that reduces a term one step. The strategy \code|eval|
applies rule \code|Eval| repeatedly from bottom to top.

\begin{figure}
\begin{example}{~}
\begin{lstlisting}[language=stratego]
module Bool
signature
  sorts Bool
  constructors
    Not   : Bool -> Bool 
    And   : Bool * Bool -> Bool
rules
  Eval : Not(True)      -> False
  Eval : Not(False)     -> True
  Eval : And(True, x)   -> x
  Eval : And(x, True)   -> x
  Eval : And(False, x)  -> False
  Eval : And(x, False)  -> False
strategies
  eval = bottomup(repeat(Eval))
\end{lstlisting}
\label{ex:stratego-module}
\end{example}
\end{figure}

We choose Stratego to implement the transformation into first-order
formulas and the type checker generator as it interacts well with
\gls{sdf}, has facilities to integrate into tool-chains and it allows
to write abstract and generic transformations.

\subsection{Spoofax}
Spoofax~\cite{KatsV10} started with the goal to provide an \gls{ide}
for \gls{sdf} and Stratego. It was then developed into a language
workbench for Eclipse that allows language development with editor
support for both, the meta-languages and the developed
language. Spoofax allows smooth switching between both editor services
and allows to deploy the editor for the developed language
standalone. The editor services provide syntactic and semantic
analysis based on live parses, with error recovery and origin
tracking. Those facilities are implemented language pragmatic, which
allows developers to focus on language specific parts.

We use Spoofax in this thesis to provide editor support for the
specification language and as glue between \gls{sdf} and Stratego in
the development of the specification language, the formula generator
and the type checker generator.

\subsection{Vampire}
We use the automatic theorem prover Vampire~\cite{VoronkovVampire} to
prove classical first-order logic propositions about type system
specifications. Vampire is able to parse formulas in the \gls{tptp}
format and provides detailed output for the conducted proofs. We use
this output to extract information about the applied axioms, which are
in our case typing rules. However, the choice of Vampire was rather
arbitrary every first order theorem prover that has support for
\gls{tptp} will do the job. Of course the results might vary depending
on the performance of the theorem prover.

\subsection{Alternatives}
We chose Spoofax and its components, because its feature set fits well
to our goals and there was some previous related work based on
it. However there are alternative \emph{language workbenches} that
could have been a good fit as well. For example
Rascal~\cite{klint2009rascal} a meta-programming language that has
among others support for context-free grammars, algebraic data-types,
relations, relational calculus operators, advanced patterns matching,
generic type-safe traversal, comprehensions, and string templates for
code generation. The paper~\cite{workbenches} gives an overview of the
state of the art in language workbenches.

\section{Type Systems}
\textit{Type theory} started as an attempt by Gottlob Frege to solve
Russel's paradox, which shows that \naive{} set theory is
inconsistent. Frege argued that a predicate requires an object as
argument and cannot have itself as an argument, as it is the
subject. So the initial motivation for type theory was to avoid
paradoxes and contradictions in logics and rewrite systems. The term
\textit{type system} refers to type theories whose rewrite systems are
programming languages. Type systems address the problem to ensure that
programs have meaning, whereas in type theory the problem is the
ensure the consistency of a logic.

What does it mean to ``ensure that programs have meaning''? It means
that one wants to filter the useful programs. It is not useful to have
a syntactically valid program that has no semantics as this program
will behave unexpectedly at some point. A type system assigns types to
the expressions of a programming language. A \textit{type checker} can
then verify, whether the types of the programs expressions match
according to predefined typing rules. If the type checker succeeds the
program is \textit{well-typed} and has meaning, i.e.\ the program will
not misbehave due to undefined semantics. Depending on the
expressiveness of the type system also the correctness of the
computation that the program performs can be shown.

\section{First-Order Logic}

%%% Local Variables: 
%%% mode: latex
%%% TeX-master: "../report"
%%% End: 
