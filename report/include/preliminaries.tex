
\chapter{Preliminaries}
\section{Tools}
In this section the tools used in this thesis are introduced. Their
capabilities and usage is presented and it is argued for what purpose
they are used in the thesis and why they are suite for those.
\subsection{SDF}
\todo[inline]{Citations needed.}
The syntax definition formalism (SDF) is a formalism to define syntax
like the Backus-Naur-From (BNF). SDF allows to define lexical and
context-free grammars and is, in contrast to BNF, modular. This
modularity allows to compose SDF syntax definitions. SDF is also
supported by parser generators. In contrast to traditional parser
generators like YACC, SDF specifications are purely declarative, which
increases understandability of the specifications.

The modularity of SDF is a consequence of being a generalized LR
parsable. Generalized parsing means that parsing might be
indeterminate, i.e. ambiguities in the specifications are
allowed. This might not seem to be an advantage, but generalized
parsing has some nice implications. First the class of supported
grammars is bigger and therefore enforces less restrictions and allows
a natural definition of languages. Second, it enables the modularity
of SDF. The composition of normal LR grammars is not guaranteed to be
an LR grammar again, but this is guaranteed for generalized LR
grammars.

An other advantage is that SDF is also scannerless parsable, that
means that tokenization and parsing can be done in a single step.

\todo[inline]{Create and explain an small example}
\todo[inline]{Talk about layout constraints}

SDF is used in this thesis to define the syntax of the specification
language for type systems. Features like layout constraints make SDF a
good tool for defining complex syntax for natural deduction rules. The
modularity and composability make it possible to define that syntax of
the specification language independent of the language under
consideration.
\subsection{Stratego/XT}
Stratego is a framework for the development of transformation
systems. It consists of the transformation language Stratego and a set
of tools XT for tasks like parsing and pretty printing. The paradigm
of Stratego is to use user-definable strategies for
rewriting. In Stratego the following abstraction levels can be
distiguished.

\begin{description}
\item[Transformation rules] are basic rewrite rules on the structure
  of the abstract syntax tree (AST).
\item[Transformation strategies] are the glue between the
  transformation rules. They combine rules and define the order of
  application and the traversal order of the AST. These strategies can
  be defined generically. Context information are passed using
  \textit{scoped dynamic rewrite rules} during the traversal, as
  normal rewrite rules are context-free.
\item[Transformation tools] allow to compile transformation strategies
  into a stand-alone program. The interface between such programs is
  the ATerm format for ASTs.
\item[Transformation systems] describe a set of programs created by
  the transformation tools, that form a source-to-source
  transformation usually including a parser and pretty printer.
\end{description}

\todo[inline]{Create and explain an small example}

Stratego interacts well with SDF and has facilities to integrate
nicely into tool-chains. In addition it allows to write abstract and
generic transformation. That makes it a good choice for the
transformation system.
\subsection{Spoofax}
\todo[inline]{Citations needed.}
Spoofax started with the goal to provide an IDE for SDF and
Stratego. It was then developed to a language workbench for Eclipse
that allows language development with editor support for both, the
languages used for development and the developed language. Smooth
switching between both editor services. The editor for the developed
language can be deployed standalone. Those editor services provide
syntactic and semantic analysis based on live parses, with error
recovery and origin tracking. Those facilities are implemented
language pragmatic, which allows developers to focus on language
specific parts.

Spoofax is used in this thesis to provide editor support for the
specification language and as glue between SDF and Stratego in the
development of the specification language, the formula generator and
the type checker generator.
\subsection{Alternatives}
\section{Type Systems}
\textit{Type theory} started as an attempt by Gottlob Frege to solve Russel's
paradox, that shows that \naive set theory is inconsistent. Frege
argued that a predicate requires an object as argument and cannot have
itself as an argument, as it is the subject.\todo{Make this more
  precise} So the initial motivation for type theory was to avoid
paradoxes and contradictions in logics and rewrite systems. The term
\textit{type system} refers to type theories whose logics rewrite
systems are programming languages. Type systems address therefore
similar problems as general type theory. The problem they address is
to ensure that programs have meaning, whereas in type theory the
problem is the ensure the consistency of a logic.

What does it mean to ``ensure that programs have meaning''. First, it
means that one wants to filter the useful programs. It is not useful
to have a syntactically valid program that has no semantics. This is
done by assigning types to the expressions of the program. A
\textit{type checker} can then check, whether those types match with
the expressions. If the type checker succeeds the program is
\textit{well-typed} and has meaning. Second, a type system allows to
establish abstractions through user definable data types. When type
checking a program, it is also ensured that those abstractions are not
violated.\footnote{This is only true for so called ``strong'' type
  systems.} In this thesis \textit{static} type systems are
considered. In a static type system types are checked prior to
execution. This early checks can provide guarantees about the programs
behavior 
\section{First-Order Logic}

%%% Local Variables: 
%%% mode: latex
%%% TeX-master: "../report"
%%% End: 
