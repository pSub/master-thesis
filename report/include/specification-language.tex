\chapter{Specification Language}
\label{cha:spec-lang}
Our type system specification language is a \gls{dsl} for people
familiar with type systems or type theory. While developing it, we had
the following goals in mind. The specification language should be

\begin{itemize}
\item close to text-book formalisms used in the type system community
\item purely declarative
\item modular
\item usable with existing syntax definitions of programming languages
\end{itemize}

Those characteristics fit well with the goals of making it easy to
experiment with type systems and to create type systems from a
high-level specification.
\section{Language Design}
\label{sec:design--architecture}
In this sections we introduce the architecture of the specification
language with references to a small examples and argue how this
architecture reflects the goals mentioned above.

Specifications in our specification language are divided into eight
sections, of which two are optional. In a specification the sections
have to be present in the order they are introduced here.

\paragraph{Module name} The first section declares the name of the
specification module. This name is an unique identifier and used for
imports. The module name may contain an arbitrary combination of
numbers, letters and the symbols \code|.| (dot), \code|-| (dash) and
\code|/| (slash).

\begin{example}{}
\begin{lstlisting}[language=sltc]
module target-language/typesystem
\end{lstlisting}
\label{ex:module-section}
\end{example}

Example~\ref{ex:module-section} shows the declaration of a module with
the unique identifier \code|target-language/typesystem|.

\paragraph{Imports} The second section declares which other type
system specifications are imported and is optional. It is possible to
prevent sections of a module from being imported with the keyword
\code|hiding|.

\begin{example}{~}
\begin{lstlisting}[language=sltc]    
imports some-specification hiding (language contexts)
        another-specification
\end{lstlisting}
\label{ex:import-section}
\end{example}

The specification in Example~\ref{ex:import-section} imports the
specification module \code|some-specification| without the sections
\code|language| and \code|contexts| and it imports the module
\code|another-specification|.

\paragraph{Language} The third section declares the language for which
a type system should be specified. This language needs to be present
as a \gls{sdf} module located at the specified path. We call this
language from now on \emph{target language}. If other modules are
imported the target language must be defined only once in the closure
of the module.

\begin{example}{~}
\begin{lstlisting}[language=sltc]
language specifications/simply-typed-lambda-calculus/syntax
\end{lstlisting}
\label{ex:language-section}
\end{example}

In Example~\ref{ex:language-section} the \gls{sdf} definition at
\code|specifications/simply-typed-lambda-calculus/syntax.sdf| is used;
This path is relative to the \code|syntax| folder of the Spoofax
project. A \gls{sdf} definition file (a file with the suffix
\code|def|) located at the same path needs to be present for analysis
of the target language, see Section~\ref{sec:constr-templ-optim}.

\paragraph{Contexts} The fourth section declares the contexts that can
be used in the judgments and rules. In specifications contexts are
used to track or collect information during type checking. Informally
these contexts are declared as cross-products of non-terminals, i.e.\
a context is a set consisting of tuples of terminals that can be
produced by the non-terminals in the declaration. Every context has a
name, which can be used as a non-terminal in the specification.

Contexts serve in most use cases as bindings for variables. Therefore
it is necessary to be able to look up terminals in a context. In the
context declaration each non-terminal is annotated with an input
(\code|{I}|) and output (\code|{O}|) tag. Those tags specify the key and
value positions of the context. Depending on those annotations we
generate look up functions. If all non-terminals are tagged as inputs
the generated look up function will perform a normal membership
test. However, a context with only output tags is not useful as it is
not possible to add elements to it.

Given a context declaration \code|Z := A{I} x B{I} x C{O} x D{O}| an
instance of this context looks like \code|(a : b : c : d) ; z| where
\code|a|, \code|b|, \code|c| and \code|d| are terminals produced from
the non-terminals \code|A|, \code|B|, \code|C|, \code|D|,
respectively. \code|z| is an other instance of context \code|Z| or the
empty context, which is written \code|()|.

\begin{example}
\begin{lstlisting}[language=sltc]
contexts Binding := ID{I} x Type{O}
\end{lstlisting}
\label{ex:context-section}
\end{example}

Example~\ref{ex:context-section} shows a context called \code|Binding|
that consists of a non-terminal \code|ID| tagged as input and a
non-terminal \code|Type| tagged as output. This contexts models a type
binding for identifiers. We generate from this context declaration a
look up function of the form \code|i : t in c| where \code|i|
represents a terminal produced from \code|ID|, \code|t| a terminal
produced from \code|Type| and \code|c| represents an instance of
context \code|Binding|. 

\paragraph{Meta-variables} The fifth section declares
meta-variables. In specifications meta-variables are used to refer to
expressions of the target language and to contexts. The declaration of
a meta-variable consists of a class, a prefix and a set of
non-terminals from the target language and from the specification. We
use the class to distinguish different kinds of
meta-variables. Currently we use classes only for merging
meta-variable declarations when resolving imports. For details see
Section~\ref{sec:generate-sdf}.

The set of non-terminals defines the scope of a meta-variable. Every
non-terminal contained in this set is extended with productions for
meta-variables. In other words, a meta-variable is a substitute for
every terminal that can be produced from one of the non-terminals. We
explain the extension of the target language in detail in
Section~\ref{sec:generate-sdf}.

Syntactically a meta-variable is a string of numbers and letters that
is prefixed with the prefix of the meta-variable declaration. The sole
purpose of the prefix is to avoid syntactic ambiguities. The prefix
itself can consist of numbers, letters and the following symbols
\code|~|, \code|$|, \code|%|, \code|&| and \code|?|. %% $
There is currently no way to add new symbols to that list, besides
editing the syntax definition of the specification language. Usually a
prefix is chosen such that it is not a prefix of a construct of the
target language, to reduce the chance of encountering ambiguities.

\begin{example}{~}
\begin{lstlisting}[language=sltc]
meta-variables 	Term "~" { Type Exp }
                Ctx "$" { Context }
                Id "%" { ID }
\end{lstlisting}
\label{ex:meta-variable-section}
\end{example}

Example~\ref{ex:meta-variable-section} declares three classes of
meta-variables. The first class is called \code|Term| and each
meta-variable of this class has the prefix \code|~|. Productions for
meta-variables of this class are added to the non-terminals
\code|Type| and \code|Exp|.

\paragraph{Judgments} The sixth section of a type system specification
module declares judgments. In our specification language judgments are
the basic building blocks of the rules defined in the next
section. This is common in the specification of deduction systems in
general. Judgments can be thought of as the ``syntax'' of the type
system, the semantics is defined by the rules.

A judgment can be defined rather arbitrarily from a combination of
strings of letters and numbers, non-terminals of the target language
and the names of the contexts. Those can be mixed freely, as long as a
string separates the non-terminals and context names. This restriction
is only needed to reduce the number of syntactic ambiguities in the
language. Judgments do not have a name as it is currently not possible
to refer to them in any other way than in instances of them. To
separate judgment declarations from each other, each declaration must
be finalized with a dot.

Non-terminals of the target language and context names need to be
annotated with input/output-tags in the same way as for
contexts. Those tags describe which parts of the judgment need to be
computed by rule applications and which are provided as input
parameters to the rule application. Non-terminals of contexts
currently only support the input tag.

Equality and inequality are predefined built-ins, but have to be
introduced as judgments. This enables to define precisely for which
non-terminals equality/inequality should be available and prevents
other uses at parse time. Currently it is not possible to define
(in-)equalities between contexts. Equality and inequality judgments
are defined by appending \code|is Eq| or respectively \code|is Neq| to
the judgment declaration.

\begin{example}{~}
\begin{lstlisting}[language=sltc]
judgments Context{I} "|-" Exp{I} ":" Type{O}.
          Type{I} "<:" Type{I}.
          Exp{I} "=" Exp{I} is Eq.
\end{lstlisting}
\label{ex:judgment-section}
\end{example}

Example~\ref{ex:judgment-section} shows three judgments. The first
could be the typing judgment of a variant of the simply typed lambda
calculus with a context, an expression of the target language as input
and a type of the target language as output. The second judgment
defines a relation between the types of the target language, i.e.\ it
has only input positions. This judgment could represent a subtyping
relation. The last judgment declares an equality between expressions.

\paragraph{Rules} The seventh section of the module declares the
(typing) rules. These rules define the semantics for the judgments
declared in the previous section. The syntax of the rules replicates
the form of inference rules: A (possibly empty) list of premises
separated from a conclusion by a horizontal line. Premises and
conclusion are instantiated judgments. All meta-variables that occur
free in rules are implicitly all-quantified. This means that all
meta-variables are all-quantified, as there is currently no mechanism
to bind variables in typing rules.

Rules can be annotated with a name. Rule names increase the
readability of the specification and allow to create human readable
derivation traces. The rules have also support for custom error
messages. There are two kinds of error annotations. Premises can be
annotated with \code|@error msg|, where \code|msg| can contain
meta-variables and holes (written \code|{}|) interleaved with
arbitrary strings. Those errors are thrown if the premise could not be
derived or if the calculated output does not match the expected
output. The meta-variables are instantiated with the appropriate terms
and the hole with the expected output. The other kind of errors are
prefixed with \code|@implicit| and are thrown if an implicit equality
between two meta-variables cannot be satisfied. The meta-variables of
implicit equalities are distinguished in the error message by
\code|@[number]| annotations, where \code|[number]| is a natural
number. The conclusion can also be annotated with error messages for
implicit equalities. Section~\ref{sec:constr-gener} and
Section~\ref{sec:constraint-solving} explain how error messages are
implemented in the type checker generator. Note that error messages
are always attached to the preceding premise.

\begin{example}{~}
\begin{lstlisting}[language=sltc]
judgments Context{I} "|-" Exp{I} ":" Type{O}.

rules

%x : ~T in $C @error %x "should have type" ~T "but has" {}.
============= T-var
$C |- %x : ~T

(%x : ~T1 ; $C) |- ~e : ~T2
@error ~e "should have type" ~T2 "but has" {}.
===================================== T-abs
$C |- fun %x : ~T1 (~e) : ~T1 -> ~T2

$C |- ~e1 : ~T1 -> ~T2
$C |- ~e2 : ~T1 @error ~e2 "should have type" ~T1 "but has" {}.
==================== T-app
$C |- ~e1 ~e2 : ~T2
\end{lstlisting}
\label{ex:rules}
\end{example}

Example~\ref{ex:rules} shows the rules typing rules of \gls{pcf} for
variables (\code|T-var|), function abstraction (\code|T-abs|) and
function application (\code|T-app|). Rule \code|T-var| models that
some variable \code|%x| has type \code|~T| if it has type \code|~T| in
context \code|$C|. If this check fails the annotated error message is%$
thrown, where \code|~T| is replaced by the expected type and \code|{}|
by the actual type of variable \code|%x|.

Rule \code|T-abs| in Example~\ref{ex:rules} expresses that function
with argument \code|%x| of type \code|~T1| and a function body
\code|~e| has type \code|~T1 -> ~T2| if the function body has type
\code|~T2| under a context that is extended with the function
argument. Function application is covered by rule \code|T-app|: if
\code|~e1| is a function of type \code|~T1 -> ~T2| and the type
\code|~e2| of matches its argument, then the application of \code|~e2|
to \code|~e1| has type \code|~T2|.

\begin{example}{~}
\begin{lstlisting}[language=sltc]
judgments
Context{I} "|-" Exp{I} ":" Type{O}.

rules

====================== Subst-Eq
~S = [ %x -> ~S ] %x@1    @implicit %x "does not equal" %x@1.
\end{lstlisting}
\label{ex:implicit-annotation}
\end{example}

Example~\ref{ex:implicit-annotation} shows the rule \code|Subst-Eq|
from the SystemF specification in
Appendix~\ref{appendix:systemf}. \code|Subst-Eq| has no premises and
an implicit equality annotated with an error message in the
conclusion. Here \code|%x| and \code|%x@1| refer to the same
meta-variable. The annotation \code|@1| is used to distinguish the
variables in the error message.

\paragraph{Conjectures} Tests for a specification are called
\textit{conjectures}. Their syntax is similar to the syntax of rules,
with two exceptions. It is not possible to annotate premises or
conclusions with error messages and a conjecture can be marked as
\textit{not derivable} by prepending the separating line with a
slash. Marking conjectures not derivable allows to formulate negative
tests.

\begin{example}{~}
\begin{lstlisting}[language=sltc]
============================
() |- let fac : int -> int = 
  fix f : int -> int (
    fun n : int (
      ifz n then 1 
      else n * (f (n - 1))
    )
  )
 in (fac 3) : int

/===========================
() |- fun x : int (x) : int
\end{lstlisting}
\label{ex:conjecture-section}
\end{example}

Example~\ref{ex:conjecture-section} shows two conjectures of the
\gls{pcf} implementation. They use the judgment shown in
Example~\ref{ex:judgment-section}. The first conjecture asserts that
the type of the faculty function applied to \code|3| is \code|int| and
the second conjecture asserts that the identity function for integers
(\code|fun x : int (x)|) has type \code|int| is not derivable.

\paragraph{Comments} Line comments (\code|//|) and block comments
(\code|/* ... */|) can be inserted everywhere in the module.

\subsection{Design Assessment}
How does this design reflect the characteristics from the beginning of
this chapter? We will address this point by point.

\begin{description}
\item[Usage] Formal definitions of type systems usually consist of
  judgments, rules and auxiliary definitions for contexts. All those
  components can be represented in a natural way in the specification
  language. Judgments can be defined as an arbitrary combination of
  non-terminals and separation symbols, as long as the syntax does not
  create ambiguities. Rules are written in a natural deduction style,
  because this is the most common formal representation of typing
  rules and it acknowledges the deductive nature of typing rules in
  general. Contexts can be defined using a set like notation. This is
  close to the intuitive semantics of contexts and allows to generate
  commonly used syntax.
\item[Declarative] Nothing that can be defined in the specification
  language has side effects (i.e.\ breaks referential transparency) or
  possibilities to embed executable code. Therefore the specification
  language is purely declarative, it focus on \emph{what} should be
  done rather then \emph{how} it should be done.
\item[Modularity] Type system specifications are organized in
  composeable modules. A module can import other modules or even only
  parts of other modules. This enables the reuse of existing type
  specifications and the separation of orthogonal features.
\item[Integration] Type systems can be defined for every programming
  language for which a \gls{sdf} syntax definition exists. These
  definitions can be used in most cases without further
  modifications. One of the requirements is that they introduce a
  constructor for every context-free production. Modifications are
  also needed if one wants to use a language concept inductively in
  the type system, but has not implemented this concept with explicit
  induction in the syntax. The implementation of records in the simply
  typed lambda calculus in Appendix~\ref{appendix:stlc-records} is an
  example for this.
\end{description}
\section{Implementation}
\label{sec:generate-sdf}
In this section we describe how the specification language is
implemented and how the new syntactic constructs that can be defined
in a specification are integrated.

The specification language itself is defined in \gls{sdf} and consists
of four \gls{sdf} modules. The module \code|Common| containts lexical
definitions that are used in multiple modules, the module
\code|BaseLanguage| which defines the syntax of the specification
language, the module \code|Generated| which contains new syntax for
meta-variables, contexts and jugmens that was defined in a
specification and the main module \code|SLTC| that combines the
previous modules.

The module \code|Common| only defines character classes and lexical
restrictions, therefore we will not explain it in detail.

The syntax of the specification language, which is independent of the
target language, is defined in the module \code|BaseLanguage|, and
is parameterized by the non-terminals \code|TypingJudgment| and
\code|MetaVariable|. These non-terminals depend on the actual target
language and are therefore defined in \code|Generated|.

In a new project the module \code|Generated| does not exist in the
syntax folder, as no specification is in use. Building a project
copies a dummy \code|Generated| module from the resource folder into
the syntax folder. The dummy \code|Generated| module contains empty
productions for the non-terminals \code|TypingJudgment| and
\code|MetaVariable| to ensure that a compilation of the whole project
is possible without a specification (e.g.\ to run tests). We describe
the generation of a specification specific \code|Generated| module in
the following.

The strategy \code|toSdf| transforms a specification into a \gls{sdf}
\gls{ast} which is then pretty printed and saved in the syntax
folder. The generated module \code|Generated| imports module
\code|Common| and the \gls{sdf} file of the target language. In
addition it contains context-free grammars for context, meta-variables
and judgment declarations.

For each context declaration the strategy \code|make-contexts|
generates productions for the empty context, context bindings and
context look ups. Figure~\ref{fig:context-productions} shows the
resulting productions, where \textit{Name} is the name of the context
declaration and \code|n| the position of the context declaration in
the specification.

\begin{figure}
\begin{grammar}
  <Name> ::= `ContextEmpty-n'
  \alt `ContextBind-n' <Elem> <Name>
  \alt `ContextLookup-n' <Elem> <Name>
  \alt `(' <Name> `)'

  <Elem> ::= <String> | <String> `:' <Elem>
\end{grammar}
\caption{Context productions}
\label{fig:context-productions}
\end{figure}

Figure~\ref{fig:meta-variable-productions} shows the productions
created by \code|make-variable| for meta-variables. A meta-variable
consists of its prefix, a name and an optional error annotation. There
are two kinds of productions for meta-variables. The first kind
extends every non-terminal (in the production called \textit{Scope})
listed in the meta-variable definition of the target language. The
second kind (the non-terminal \emph{MetaVariable}) enables the use of
meta-variables in error messages. In the constructor of meta-variables
\code|m| is either the context number in case \textit{Scope} is a
context and otherwise the meta-variable class.

\begin{figure}
\begin{grammar}
  <Scope> ::= `MetaVariable-m' <Prefix> <MetaVariableName> <Anno>

  <MetaVariable> ::= `MetaVariable-m' <Prefix> <MetaVariableName>
  <Anno>

  <Anno> ::= $\epsilon$ | `@' <ErrorNumber>
\end{grammar}
\caption{Meta-variable productions}
\label{fig:meta-variable-productions}
\end{figure}

Strategy \code|make-judgment| creates productions for judgment
declarations. For each judgment we generate a production that consists
of all non-terminals separated by the separators from the
declaration. The constructor of judgments is either composed from the
string \code|TypingJudgment| or in the case of a built-in from the
built-in name and the position of the judgment in the specification.

The module \code|SLTC| plugs all modules together. It imports the
module \code|Generated| and instantiates the parameters of module
\code|BaseLanguage|.

Imports are implemented using the \gls{nabl} that is integrated in
Spoofax. \gls{nabl} ensures that the module names are unique, that
imports can be resolved and annotates modules with meta-information
about modules. The Stratego strategy \code|resolve-imports| does the
actual resolving of the imports before a module gets used.

First \code|resolve-imports| fetches all module definitions that are
imported. Then it merges the fetched modules into the current
module. This results in a module that contains the declarations of the
current module plus all not excluded declarations from the imported
modules. All sections are merged separately and redundancies such as
duplicate declarations are removed.

%%% Local Variables: 
%%% mode: latex
%%% TeX-master: "../report"
%%% End: 
