\chapter{Specification Language}
Our type system specification language is a domain-specific language
for people who work on type systems or type checkers. It was created
with the following characteristics in mind:

\begin{itemize}
\item Close to the formal or mathematical notations that are commonly
  used to narrow the gap between the formal model and the
  implementation side.
\item Purely declarative to increase maintainability and readability of
  specifications.
\item Modular to combine type systems and increase reuse.
\item Use existing syntax definitions of programming with minor
  modifications.
\end{itemize}

Those characteristics fit well to the goals of making it easy to
experiment with type systems and to create type systems from a
high-level specification.
\section{Design \& Architecture}
In this sections we introduce the architecture of the specification
language with references to a small example specification and argue
how this architecture reflects the characteristics mentioned above.

A specification is divided into eight sections, of which two are
optional.

\paragraph{Modulename} The first section declares the name of the
specification module. This name is a unique identifier and used for
imports.

\paragraph{Imports} The second section declares which other
specifications are imported and is optional.

\paragraph{Language} The third section declares the language for which
a type system should be specified. This language needs to be present
as an gls{sdf} module located at the specified path. This language is
from now on called \textit{target language}.

\paragraph{Meta-variables} In the fourth section meta-variables are
declared. The declaration of a meta-variable consists of a class
\refcallout{ex:spec:class}, of a prefix \refcallout{ex:spec:prefix}
and of a set of non-terminals of the target language
\refcallout{ex:spec:scope}. The class can be used to distinguish
different kinds of meta-variables, e.g.\ for optimization
purposes. \todo{Give an example optimization}

The set of non-terminals defines the scope of the meta-variable. Every
non-terminal that is contained in this set is extended with
meta-variables. In other words, a meta-variable is a substitute for
every terminal that can be produced from the
non-terminal. Section~\ref{sec:generate-sdf} explains in detail how
the target language is extended.

Syntactically a meta-variable is a string of numbers and letters that
is prefixed with the prefix of the meta-variable definition. The sole
purpose of the prefix is to avoid ambiguities while parsing. The
prefix itself can consist of numbers, letters and the following
symbols \verb|~|, \verb|$|, \verb|%|, \verb|&| and \verb|?|.
\todo{Is this list complete?} Usually a prefix is chosen such that it
is not a prefix of a construct of the target language.

\paragraph{Contexts} In the fifth section all contexts that will be
used in the judgments are defined. Informally these contexts are
declared as cross-products of non-terminals, i.e. an context is a set
consisting of pairs of terminals that can be produced by the
non-terminals in the declaration. Every context has a name, which can
be used in further declarations. Lookups may differ for contexts and
use cases, therefore one has to annotate the non-terminals with input
and output tags. Terminals at positions that are tagged with input are
used to look up terminals tagged with output that are saved in the
context. An context for which all non-terminals are tagged with input,
is equivalent to a normal look up in a set and will return whether the
tuple could be found. An context with only outputs acts as a sink, one
can put things into it, but never retrieve anything back.

\paragraph{Judgments} The sixth section declares type judgments. Type
judgments can be declared in a flexible way, non-terminals can be
mixed arbitrarily with separators. This allows the programmer\todo{Is
  this the right word?} to declare the type judgments as close to
formalizations as possible. As for contexts the non-terminals in the
type judgments have to be annotated with input/output-tags to
determine what should be computed.

\paragraph{Rules/Conjectures} In the last two sections the typing
rules and conjectures are declared. Both typing rules and conjectures
have the same syntax. Typing rules specify the type system and
conjectures are used to provide test cases and are optional. A typing
rule consists of a possible empty list of typing judgments, an
optional name and a typing judgment as conclusion.

\begin{figure}
\scriptsize
\begin{verbatim}
module example
imports common
language syntax/simply-typed-lambda-calculus
meta-variables 	Term !callout`ex:spec:class' "~" !callout`ex:spec:prefix' { Type Exp } !callout`ex:spec:scope'
                Ctx "$" { Context }
                Id "%" { ID }
contexts Context := ID{I} x Type{O}
judgments Context{I} "|-" Exp{I} ":" Type{O}.
rules

%x : ~T in $C @error %x "should have type" ~T "but has" INFTYPE
============== T-Var
$C |- ~x : ~T

(%x : ~T ; $C) |- ~t : ~T @error ~t "should have type ~T" "but has" INFTYPE
================================= T-Abs
$C |- \ %x : ~T . ~t : ~T -> ~T

$C |- ~t1 : ~T11 -> ~T12 @error ~t1 "should have type" ~T11 "->" ~T12 "but has" INFTYPE
$C |- ~t2 : ~T2          @error ~t2 "should have type" ~T2 "but has" INFTYPE
========================= T-App
$C |- ~t1 ~t2 : ~S
\end{verbatim}
\caption{Specification for a type system for the simply typed lambda calculus}
\end{figure}

How does this design reflect the goals declared above?

\begin{description}
\item[Usage] Mathematical definitions of type systems usually consist
  of judgment, rules and auxiliary definitions for contexts. All
  those elements can be represented in a natural way in the
  specification language. Judgments can be defined as an arbitrary, as long
  as this syntax does not create ambiguities, combination of
  non-terminals and separation symbols. Rules are written in a natural
  deduction style, because this is the most common representation for
  typing rules and acknowledges the deductive nature of typing rules
  in general. Contexts can be defined using a set like
  notation. This is close to the intuitive semantics of contexts
  and allows to generate commonly used syntax.
\item[Declarative] Everything that can be defined in the specification
  language has no side effects or possibilities to embed executable
  code.
\item[Modularity] \todo[inline]{Imports are not implemented yet.}
\item[Integration] Type systems can be defined for every programming
  language for which a SDF syntax definition exists. These definitions
  can in most cases be used in the specification language without
  modifications. Modifications are needed if one wants to use a
  language concept inductively in the type system, but has not
  implemented this concept with explicit induction in the
  syntax.\todo[inline]{Given an example.}
\end{description}
\section{Implementation}
\label{sec:generate-sdf}
\todo[inline]{Integration with sugar-lang is not done yet.}

%%% Local Variables: 
%%% mode: latex
%%% TeX-master: "../report"
%%% End: 
