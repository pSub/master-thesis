
\chapter{Specification Language}
The goal of this thesis is to create a generator for type checkers,
like there are generators for lexers and parers. Such a generator can
only be a useful tool, if the specifications from which the type
checkers are generated are easy to read and write for people who are
familiar with type systems and if the specification language is
expressible enough to handle real-world languages. As there is no
common format for type system specifications, a new specification
language was designed with the following goals in mind:

\begin{enumerate}
\item Close to the formal or mathematical notations that are commonly
  used to close the gap between the formal model and the
  implementation side.
\item Declarative, makes it easy to try things out and increases
  maintainability and readability of specifications.
\item Modularity, it should be possible to combine type systems to
  increase reuse.
\item Existing programming language definitions should integrate nicely.
\end{enumerate}
\section{Design \& Architecture}
A specification is divided into eight sections, of which two are
optional. The three sections are for organizational purposes. The
first section declare the name of the specification. This name is a
unique identifier and used for imports. The second section declares
which other specifications are imported. The third section declares
the language for which a type system should be specified. This
language needs to be present in SDF and there must be a module with
exactly that name. \todo[inline]{Does it make sense to allow imported
  files to declare a language?}

In the fourth section meta-variables are declared. The declaration of
a meta-variable consists of an identifier, of a prefix and of a set of
non-terminals of the base language. The identifier can be used to
distinguish different kinds of meta-variables.\todo{Currently this is
  not used} A meta-variable can consist of letters and numbers, to
ensure that we can parse meta-variables of different kinds
unambiguously every meta-variable must be prepended with the declared
prefix. As we want to introduce meta-variables at different positions
in the base language we need a way to extend the syntax of the base
language. The set of non-terminals in a meta-variable definition lists
the non-terminals that shall be extended.

\todo[inline]{Make up my mind about the name of those environments}
In the fifth section all environments that will be used in the
judgments are defined. Informally these environments are declared as
cross-products of non-terminals, i.e. an environment is a set
consisting of pairs of terminals that can be produced by the
non-terminals in the declaration. Every environment has a name, which
can be used in further declarations. Lookups may differ for
environments and use cases, therefore one has to annotate the
non-terminals with input and output tags. Terminals at positions that
are tagged with input are used to look up terminals tagged with output
that are saved in the environment. An environment for which all
non-terminals are tagged with input, is equivalent to a normal look up
in a set and will return whether the tuple could be found. An
environment with only outputs acts as a sink, one can put things into
it, but never retrieve anything back.

The sixth section declares type judgments. Type judgments can be
declared in a flexible way, non-terminals can be mixed arbitrarily
with separators. This allows the programmer\todo{Is this the right
  word?} to declare the type judgments as close to formalizations as
possible. As for environments the non-terminals in the type judgments
have to be annotated with input/output-tags to determine what should
be computed.

In the last two sections the typing rules and conjectures are
declared. Both typing rules and conjectures have the same
syntax. Typing rules specify the type system and conjectures are used
to provide test cases and are optional. A typing rule consists of a possible empty list
of typing judgments, an optional name and a typing judgment as
conclusion.

\begin{figure}
\begin{verbatim}
module example
imports common
language simply-typed-lambda-calculus
meta-variables 	Term "~" { Type Exp }
                Ctx "$" { Context }
                Id "%" { ID }
contexts Context := ID{I} x Type{O}
judgments Context{I} "|-" Exp{I} ":" Type{O}.
rules

%x : ~T in $C
============== T-Var
$C |- ~x : ~T

(%x : ~T ; $C) |- ~t : ~T
================================= T-Abs
$C |- \ %x : ~T . ~t : ~T -> ~T

$C |- ~t1 : ~T11 -> ~T12
$C |- ~t2 : ~T2
========================= T-App
$C |- ~t1 ~t2 : ~S
\end{verbatim}
\caption{Specification for a type system for the simply typed lambda calculus}
\end{figure}
\section{Implementation}

%%% Local Variables: 
%%% mode: latex
%%% TeX-master: "../report"
%%% End: 
