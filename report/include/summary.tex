\chapter{Summary}
\section{Conclusion}
\section{Future Work}
- Implement contexts using normal key-value lists
- Implement a nice context interface
- Convert everything to sdf3
- Support multiple languages without manual recompilation
- Make syntax of typing rules more flexible (e.g. dot notation)
- Modular context descriptions like in Typmix
- contexts in error messsages
- Better caching techniques for proofs
\section{Related Work}
\subsubsection{Ott}
\subsubsection{The Synthesizer Generator}
\subsubsection{JavaCOP: Declarative Pluggable Types for Java}
JavaCOP \cite{Markstrum:2010:JDP:1667048.1667049} is a framework for
pluggable type systems in Java. It hooks directly into \code|javac|
and therefore integrates nicely into the normal development
cycle. JavaCOP provides three tools: A declarative language to
describe structural constraints on the \gls{ast} of Java programs in a
flow-insensitive manner, an API to use the declarative language for
flow-sensitive data flow analysis and a test harness which helps to
test that a program that is well-typed actually satisfies the
invariants of the type system. In contrast to~\cite{bergan2007typmix}
the syntax of Java cannot be extended.

\todo[inline]{Compare with my work}

\subsubsection{A Generator for Type Checkers}
Gast introduces in~\cite{gast2005generator} a type checker generator
that can produce type checkers from declarative type system
specifications for functional as well as imperative and
object-oriented programming languages. Type checking is done by a
specialized proof search, which is based on unification and
backtracking. A distinguishing feature of Gast's work is the
possibility to annotate the typing rules with optimizations. For
example it is possible to reuse (potentially incomplete) proofs using
subproof extraction. Gast also provides a formal foundation for his
proof search. However the resulting type checkers do not report
specialized error messages on ill-typed programs.

\subsubsection{Automatic Generation of Object-Oriented Type Checkers}
Ortin et al.\ present in~\cite{ortin2014automatic} the framework TyS
for the implementation of type checkers for object-oriented
programming languages. TyS provides a type checker constructor TyCC
which produces a type checker when given a file that specifies the
types together with the subtyping relation and operations on these
types. Operations on types correspond to typing rules and are
implemented using an API that is provided by TyS. Currently only Java
is supported for the implementation of the operations. The addition of
new languages requires the replication of the API in that language.

TyS can be used in conjunction with different parser generator tools
and has been tested with flex, bison, yacc and ANTLR. It supports
the generation of static and dynamic type checkers and was applied
(among others) to the object-oriented language Drill and the
imperative language Frog. TyS does not support polymorphic types.

In contrast to the present work the TyS framework does not provide a
high-level specification language in favor of readable generated code
and is tied to object oriented languages. It however delivers a tool
that has been successfully integrated in existing tool chains.
\subsubsection{Typmix: A Framework for Implementing Modular Extensible Type
  Systems}
Typmix~\cite{bergan2007typmix} is a framework for implementing type
systems for the extensible compiler xoc. Although type systems can be
implemented in xoc directly, the author claims that they are often
verbose and contain redundancies. Type system specifications in Typmix
are written in two languages. One describes the context modifications,
which are called scoperules. The other describes typing rules using
premisses and conclusions. The contexts in the typing rules are
referred to as \code|Env| and concrete contexts as
\code|Env.Ctx|. This separation of concerns increases modularity,
e.g.\ adding a context to a judgment requires only changes were it is
actually used. Typmix is used to implement a type system for an
ML-like language and for FeatherweightJava.

As typmix is integrated in an extensible compiler, its focus is on
extending type systems. Although the scope and type language are
declarative it has no interface to proof assistants.

\subsubsection{Automatic Type Inference via Partial Evaluation}
Tomb and Flanagan~\cite{tomb2005automatic} use Prolog to implement
type inference. If typing rules are syntax directed Prolog can type
check a program efficiently. A Prolog type checker can be easily
changed to infer types by leaving type variables unbound. However this
can easily diverge due to Prologs depth-first search. This
inefficiency is solved in~\cite{tomb2005automatic} by a two phase
approach. In the first phase some Prolog clauses are evaluated, which
is determined by a partitioning parameter. This parameter is usually
set such that relations that depend on types associated with type
variables are delayed into the second phase. These two phases resemble
a constraint generation and a constraint solving phase. The result of
the first phase may contain arbitrary Prolog terms, but the authors
claim that for common cases, the result is simple and can be
efficiently solved by e.g. Datalog. This approach also applies to
non-syntax directed typing rules but might be less efficient.

Just as the present work~\cite{tomb2005automatic} generates a
two-phase constraint-based type inference system from a high-level
specification. The advantage of choosing a logic language like Prolog
is that the correctness of the type inference algorithms is entailed
by the correctness of the partial evaluator. However the current
approach does only work for language definitions in Prolog and does
not attempt to reduce the non-determinism that is introduced from
non-syntax directed rules.

%%% Local Variables: 
%%% mode: latex
%%% TeX-master: "../report"
%%% End: 
