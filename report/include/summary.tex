\chapter{Summary}
\section{Conclusion}
\section{Future Work}
- Implement contexts using normal key-value lists
- Implement a nice context interface
- Convert everything to sdf3
- Support multiple languages without manual recompilation
- Make syntax of typing rules more flexible (e.g. dot notation)
\section{Related Work}
\subsubsection{Ott}
\subsubsection{The Synthesizer Generator}
\subsubsection{A generator for Type Checkers}
\subsubsection{Automatic Generation of Object-Oriented Type Checkers}
Ortin et al.\ present in~\cite{ortin2014automatic} the framework TyS
for the implementation of type checkers for object-oriented
programming languages. TyS provides a type checker constructor TyCC
which produces a type checker given a file that specifies the types
together with the subtyping relation and operations on these
types. Operations on types correspond to typing rules and are
implemented using an API that is provided by TyS. Currently only Java
is supported for the implementation of the operations and the addition
of new languages requires the replication of the API in that
language.

TyS can be used in conjunction with different parser generator tools
and has been tested with flex, bison, yacc and ANTLR. It does support
the generation of static and dynamic type checkers and was applied
(among others) to the object-oriented language Drill and the
imperative language Frog. TyS does not support polymorphic types.

In contrast to the present work the TyS framework does not provide a
high-level specification language in favor of readable generated code
and is tied to object oriented languages. It however delivers a tool
that has been successfully integrated in existing tool chains.
\subsubsection{A Framework for Implementing Modular Extensible Type
  Systems}
\subsubsection{Automatic Type Inference via Partial Evaluation}

%%% Local Variables: 
%%% mode: latex
%%% TeX-master: "../report"
%%% End: 
