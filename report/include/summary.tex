\chapter{Summary}
\label{cha:summary}
\section{Conclusion}
In this thesis we have presented an optimizing type checker generator
that optimizes declarative type system specifications according to
automatically conducted proofs.

We have introduced a high-level declarative type system specification
language that can be used with most \gls{sdf} syntax specifications of
programming languages. This specification language allows to define
type systems modular and close to the notation of text books. Further
we have developed a translation of type system specifications into
equivalent first-order formulas and have shown that those are suitable
for type checking using automated theorem provers.

Based on the type system specification language and the first-order
model for specifications, we have developed an optimizing type checker
generator. Our type checker generator is modular, constraint based and
uses a normalized intermediate representation of the
specifications. With the optimization phase of the type checker
generator we make a step towards to generation of efficient and syntax
directed type checkers from non-syntax directed high-level
specifications.

\section{Future Work}
In this thesis we build the foundation to create more sophisticated
optimizing type checkers. The main future goal is twofold. On the one
hand we want to develop more strategies to optimize type system
specifications, for example to detect redundancies between strictly
when-ambiguous templates. On the other hand we want to investigate
into a more expressive and efficient way of proving the properties
needed for the optimization strategies.

Besides that, it is desirable to extend the specification language
with constructs to quantify over syntactic constructs. This would
allow for example to express that all elements of a record are well
typed without explicit recursion. Another improvement of the
specification language would be a more modular way of specifying
contexts. For example inspired by Typmix~\cite{bergan2007typmix}.

Furthermore, there is room for improvements with regard to
usability. Currently a project hast to be recompiled on changes of
meta-variables, contexts, and judgments. SugarJ\cite{erdweg2011sugarj}
provides syntax definition on the fly that could solve this
problem. Another usability concern is the lack of editor integration,
for example the highlighting of type errors in the program.

To improve the performance of the type checker it would be useful to
explore different context implementations and to implement or use a
state of the art constraint solver.

%%% Local Variables: 
%%% mode: latex
%%% TeX-master: "../report"
%%% End: 
