\chapter{Evaluation}
\label{cha:evaluation}
In this chapter we present three type system specifications, discuss
how the specification language affected the formulation of typing
rules, and which optimization strategies are successful.
\section{SytemF}
\label{sec:sytemf}
We have implemented a version of the polymorphic lambda calculus
SystemF that is close to the version described in Figure 23-1 in
~\cite{Pierce:2002:TPL:509043}. The complete implementation can be
found in Appendix~\ref{appendix:systemf}. There are three notable
differences between the version in~\cite{Pierce:2002:TPL:509043} and
our implementation, which we will describe in the following.

In~\cite{Pierce:2002:TPL:509043} term and type variable bindings are
collected in a single context. We need two separate contexts, one for
the term variable binding and one for the type variable binding,
because we can only define homogeneous contexts.

\begin{lstlisting}[language=sltc]
contexts
TermBinding := ID{I} x Type{O}
TypeBinding := ID{I}
\end{lstlisting}

Context \code|TermBinding| is the term variable binding and associates
identifier with types. \code|TypeBinding| binds type variables. Type
variables are associated to nothing because we have no notion of kinds
in SystemF . In our implementation variable and type identifiers are
build from the same set of identifiers.

The typing judgment has to reflect that we have two
contexts. Therefore our typing judgment is defined as follows:

\begin{lstlisting}[language=sltc]
judgments
TermBinding{I} "|" TypeBinding{I} "|-" Exp{I} ":" Type{O}.
\end{lstlisting}

The second difference is that in~\cite{Pierce:2002:TPL:509043} it is
assumed ``that the names of (term and type) variables should be chosen
so as to be different from all names already bound'' by the
context. We enforce this by defining an explicit freshness check in
the type system specification for term and type variables. The
freshness check is implemented in a separate module, which is imported
in the specification of SystemF. In the following we show the
judgments and rules of the freshness check.
\newpage
\begin{lstlisting}[language=sltc]
judgments 
ID{I} "fresh in" TermBinding{I}.
ID{I} "fresh in" TypeBinding{I}.
ID{I} "!=" ID{I} is Neq.

rules

========================= Fresh-Term-Empty
%x fresh in (TermBinding)

========================= Fresh-Type-Empty
%x fresh in (TypeBinding)

%x != %y
%x fresh in $C
========================== Fresh-Term-Step
%x fresh in (%y : ~T ; $C)

%x != %y
%x fresh in ?C
===================== Fresh-Type-Step
%x fresh in (%y ; ?C)
\end{lstlisting}

The last difference between the text book version and our
implementation is of the same kind as the last difference. In the text
book version type substitution is assumed to be defined outside of the
type system. We had to implement type substitution within our
specification language as we have no built-in substitution
mechanism. However type substitution is implemented in the same
fashion like in a functional programming language with pattern
matching. Substitution is also implemented as a separate module and
then imported into the SystemF specification.

\begin{lstlisting}[language=sltc]
judgments
Type{O} "= [" ID{I} "->" Type{I} "]" Type{I}.
ID{I} "!=" ID{I} is Neq.

rules

===================== Subst-Eq
~S = [ %x -> ~S ] %x@1    @implicit %x " does not equal " %x@1.

%y != %x
==================== Subst-Neq
%y = [ %x -> ~S ] %y

~U = [ %x -> ~S ] ~T
========================================== Subst-All
(all %y . ~U) = [ %x -> ~S ] (all %y . ~T)

====================== Subst-Int
int = [ %x -> ~S ] int

~U1 = [ %x -> ~S ] ~T1
~U2 = [ %x -> ~S ] ~T2
==================================== Subst-Arrow
~U1 -> ~U2 = [ %x -> ~S ] ~T1 -> ~T2
\end{lstlisting}

This demonstrates that our specification language is well suited to
express standard type systems in a way close to text books as only one
difference in the implementation is due to a restriction of our
specification language. The other differences occur natural as we had
to define concepts explicitly that were left implicit before.

The generated templates for our SystemF specification show that there
is only one ambiguity, namely between \code|Subst-Eq| and
\code|Subst-Neq|. This ambiguity cannot be solved as we would have to
decide the equality of terms, before actually knowing these
terms. Nevertheless the creation of forks in the template optimization
phase is helpful. As we have only one fork, we know that all templates
besides \code|Subst-Eq| and \code|Subst-Neq| are syntax directed.
\section{Lambda-Calculus with Subtyping}
\label{sec:lambda-calculus-with}
We have implemented a variant of the simply typed lambda calculus with
records and subtyping as described
in~\cite{Pierce:2002:TPL:509043}. The specification for this type
system is divided into two modules. One module specifies the type
system without subtyping and the other module extends the first module
with subtyping. The implementation of the type system without
subtyping differs in two aspects from the text book
formalization. First, we have to implement the freshness condition
explicitly, as in the previous section. Second, we have to model rules
that talk about all elements of a record inductively.

Formula~\ref{formula:record-typing} shows the
formalization of the record typing rule
from~\cite{Pierce:2002:TPL:509043}. This rule says that a record is
well-typed if all its elements are
well-typed. In~\ref{formula:record-typing} this is expressed by
quantification over the elements of the record.

\begin{align}
\inferrule[]
{\text{for each} \; i \; \Gamma \vdash t_i : T_i}
{\Gamma \vdash \{l_i = t_i \ ^{i \in 1\dots n}\} : \{ l_i : T_i \ ^{i
    \in 1\dots n} \}}
\label{formula:record-typing}
\end{align}

This quantification is not possible in our specification
language. Therefore we model this condition inductively as shown in
the following.

\begin{minipage}[b]{.30\linewidth}
\begin{lstlisting}[language=sltc]
============= base
$C |- {} : {}
\end{lstlisting}
\end{minipage}
\begin{minipage}[b]{.65\linewidth}
\begin{lstlisting}
$C |- ~e : ~T
$C |- { $R } : { $S }
===================================== step
$C |- { %l = ~e $R } : { %l : ~T $S }
\end{lstlisting}
\end{minipage}

Here we ensure with rule \code|step| that the first element is
well-typed and the record without the first element is well
typed. Rule \code|base| is the base case for this definition and
assigns the empty record the empty record type. Note that we support
in contrast to~\cite{Pierce:2002:TPL:509043} the empty record. We
included the empty record in our definition to demonstrate a top rule
for records in the subtyping module. We have implemented the membership
test for the projection typing rule \code|T-proj| in a similar way.

The module implementing subtyping for this lambda calculus contains a
typing rule for function application that can deal with subtyping and
the definition of a subtyping relation. We have added the typing rule
for function application in favor of a general subsumption rule,
because we do not have optimization strategies to inline the
subsumption rule into the function application rule. We are able to
detect that the function application rule with subtyping is more
general (in the case of an reflexive subtyping relation) than the
function application rule without subtyping and therefore are able to
remove the latter.

The subtyping relation is defined by a generic reflexivity rule
\code|S-refl| and a rule for function types \code|S-arrow| such that
the argument type is contravariant and the return type is
covariant.

\begin{minipage}{.3\linewidth}
\begin{lstlisting}[language=sltc]
======== S-refl
~S <: ~S
\end{lstlisting}
\end{minipage}
\begin{minipage}{.6\linewidth}
\begin{lstlisting}[language=sltc]
~T1 <: ~S1
~S2 <: ~T2    
======================== S-arrow 
~S1 -> ~S2 <: ~T1 -> ~T2
\end{lstlisting}
\end{minipage}

As we have seen in Section~\ref{sec:when-ambiguities} rule
\code|S-refl| is non-syntax directed and can be safely replaced by the
following two rules.

\begin{minipage}{.3\linewidth}
\begin{lstlisting}[language=sltc]
==========
int <: int
\end{lstlisting}
\end{minipage}
\begin{minipage}{.3\linewidth}
\begin{lstlisting}[language=sltc]
{ R } = { S }
==============
{ R } <: { S }
\end{lstlisting}
\end{minipage}

The subtyping relations have a further rule that defines the empty
record \code|{}| to be the top element of records, as well as rules
for with and depth subtyping and permutation of record elements. In
conclusion we can specify a variant of the simply typed lambda
calculus with an intuitive subtyping relation and reduce the
non-determinism in the type system.

%%% Local Variables: 
%%% mode: latex
%%% TeX-master: "../report"
%%% End: 
