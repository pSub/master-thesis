\chapter{Evaluation}
\label{cha:evaluation}
In this chapter we present three type system specifications, discuss
how the specification language effected the formulation of typing
rules, and which optimization strategies are successful.
\section{SytemF}
\label{sec:sytemf}
We have implemented a version of the polymorphic lambda calculus
SystemF that is close to the version described in Figure 23-1 in
~\cite{Pierce:2002:TPL:509043}. The complete implementation can be
found in Appendix~\ref{appendix:systemf}. There are three notable
differences between the version in~\cite{Pierce:2002:TPL:509043} and
our implementation, that we describe in the following.

In~\cite{Pierce:2002:TPL:509043} term and type variable bindings are
collected in a single context. We need two separate contexts, one for
the term variable binding and one for the type variable binding,
because we can only define homogeneous contexts.

\begin{lstlisting}[language=sltc]
contexts
TermBinding := ID{I} x Type{O}
TypeBinding := ID{I}
\end{lstlisting}

Context \code|TermBinding| is the term variable binding and associates
identifier with types. \code|TypeBinding| binds type variables,
because we have no notion of kinds in SystemF type variables are
associated to nothing. In our implementation variable and type
identifiers are build from the same set of identifiers.

The typing judgment has to reflect that we have two
contexts. Therefore our typing judgment is defined as follows.

\begin{lstlisting}[language=sltc]
judgments
TermBinding{I} "|" TypeBinding{I} "|-" Exp{I} ":" Type{O}.
\end{lstlisting}

The second difference is that in~\cite{Pierce:2002:TPL:509043} it is
assumed ``that the names of (term and type) variables should be chosen
so as to be different from all names already bound'' by the
context. We enforce this by defining an explicit freshness check in
the type system specification for term and type variables. The
freshness check is implemented in a separate module, which is imported
in the specification of SystemF. In the following we show the
judgments and rules of the freshness check.

\begin{lstlisting}[language=sltc]
judgments 
ID{I} "fresh in" TermBinding{I}.
ID{I} "fresh in" TypeBinding{I}.
ID{I} "!=" ID{I} is Neq.

rules

========================= Fresh-Term-Empty
%x fresh in (TermBinding)

========================= Fresh-Type-Empty
%x fresh in (TypeBinding)

%x != %y
%x fresh in $C
========================== Fresh-Term-Step
%x fresh in (%y : ~T ; $C)

%x != %y
%x fresh in ?C
===================== Fresh-Type-Step
%x fresh in (%y ; ?C)
\end{lstlisting}

The last difference between the text book version and our
implementation is of the same kind as the last difference. In the text
book version type substitution is assumed to be defined outside of the
type system. We had to implement type substitution within our
specification language as we have no built-in substitution
mechanism. However type substitution is implemented in the same
fashion like in a functional programming language with pattern
matching. Substitution is also implemented as a separate module and
imported into the SystemF specification.

\begin{lstlisting}[language=sltc]
judgments
Type{O} "= [" ID{I} "->" Type{I} "]" Type{I}.
ID{I} "!=" ID{I} is Neq.

rules

===================== Subst-Eq
~S = [ %x -> ~S ] %x@1    @implicit %x " does not equal " %x@1.

%y != %x
==================== Subst-Neq
%y = [ %x -> ~S ] %y

~U = [ %x -> ~S ] ~T
========================================== Subst-All
(all %y . ~U) = [ %x -> ~S ] (all %y . ~T)

====================== Subst-Int
int = [ %x -> ~S ] int

~U1 = [ %x -> ~S ] ~T1
~U2 = [ %x -> ~S ] ~T2
==================================== Subst-Arrow
~U1 -> ~U2 = [ %x -> ~S ] ~T1 -> ~T2
\end{lstlisting}

This demonstrates that our specification language is well suited to
express standard type systems in a way close to text books as only one
difference in the implementation is due to a restriction of our
specification language. The other differences occur natural as we had
to define concepts explicitly that were left implicit before.

The generated templates for our SystemF specification show that there
is only one ambiguity, namely between \code|Subst-Eq| and
\code|Subst-Neq|. This ambiguity cannot be solved as we would have to
decide the equality of terms, before actually knowing these
terms. Nevertheless the creation of forks in the template optimization
phase is helpful. As we have only one fork, we know that all templates
besides \code|Subst-Eq| and \code|Subst-Neq| are syntax directed.
\section{Lambda-Calculus with Subtyping}
\label{sec:lambda-calculus-with}
\section{Information Flow Security Type System}
\label{sec:inform-flow-secur-1}

%%% Local Variables: 
%%% mode: latex
%%% TeX-master: "../report"
%%% End: 
