\chapter{Formula Generation}
\section{Goals}
\section{Why First-Order Logic?}
\subsection{Expressiveness}
\subsection{Tool-Support}
\section{Translations}
For the generation of first-order formulas not all sections of a
specification are important. The section that declares the
meta-variables and the section that declares the judgments are not
used directly at all. With regard to formula generation they are only
used to ensure correct syntax.

\todo[inline]{Implement imports and describe how they are handled.}

Environments declarations are translated in a uniform way. For every
environment two formulas are generated. These two formulas handle the
look-up of entries in the environment. One formula for the base case
and one for the step case of the lookup.

\begin{multline}
  \forall e, x_1, \dots, x_n, y_1, \dots, y_m . \\ (lookup_e(x_1,\dots,x_n,y_1,\dots,y_m, bind(x_1,\dots,x_n,y_1,\dots,y_m,e)))
\end{multline}

The above formula represents the base case of environment
look-ups. All non-terminals in the environment definition tagged with
input are translated into variables $x_1 \dots x_n$ and all tagged
with output are translated into variables $y_1 \dots y_m$
respectively. The variable $e$ ranges over all possible
environments. The $bind$ predicate used in this formula allows to add
new input/output pairs to the environment. In the implementation the
order of arguments is a bit different, but this difference is for the
discussion of the concept negligible. For details see the next
section.\todo{Use a reference} The predicate $lookup$ that is defined
also takes input/output pairs and an environment. In the base case the
environment given to $lookup$ contains the input/output pairs of
$lookup$ directly. Therefore those are trivially contained in the
environment.

\todo[inline]{Format look-up step formula nicely.}
\begin{multline}
  \forall e, x_1, \dots, x_n, x_1', \dots, x_n', y_1, \dots, y_m, y_1', \dots,
  y_m' . \\
  (lookup(x_1, \dots, x_n, y_1, \dots, y_m, e) \implies \\ lookup(x_1,
  \dots, x_n, y_1, \dots, y_m, bind(x_1',\dots,x_n',y_1',\dots,y_m',e))
\end{multline}

This formula represents the step case of environment look-ups. Its
intuition is that if some input/output pair is contained in an
environment $e$ the it is also contained in an environment in which a,
possibly different, input/output pair is stored. Variables $x_i$ and
$x_i'$ are non-terminals from the environment declaration tagged with
input, $y_i$ and $y_i'$ are non-terminals tagged with output
respectively.

The AST nodes of the programming language are translated into
predicates. To help theorem provers for each of those predicates an
axiom for injectivity and univalence is created. Those hold by
definition of the AST nodes.

The most important part is the translation of the typing
rules. Depending on whether the typing rules has premises either of
the following schema is used:

\begin{align}
  &\forall FV(c) &.\,& c \\
  &\forall FV(p_1,\dots, p_n) &.\,& p_1 \land \dots \land p_n \implies c
\end{align}

\todo[inline]{In the following free variables are actually free
  meta-variables.}
The meta-variables $p_i$ are the premises and $c$ is the conclusion of
the typing rules. The intuition of a typing rule is that the
conclusion can be derived if all premises can be derived. In terms of
first-order logic ``derived'' means for arbitrary formulas that there
is a proof and for predicates that the predicate holds under the given
interpretation. Therefore a typing rule without premises is translated
into a formula that consists of the conclusion and all-quantifies all
free variables of the conclusion. Free variables are all-quantified,
because all possible variants of the conclusion have to be derivable.
Typing rules with premises are translated into a single
implication. The premise of the implication is the conjunction of all
premises of the typing rule. This ensures that all premises need to be
derivable/satisfied. The conclusion of the implication is the
conclusion of the typing rule. This makes sense as the intuition was,
that the conclusion of the typing rule is derivable if all premises
are derivable, which is exactly the semantics of this implication.

Type judgments are translated into simple predicates. Equality and
inequality can be expressed via judgments with the annotation ``is
Eq'' and ``is Neq'', respectively. Those judgments must have exactly
two non-terminals.
\section{Implementation}
The implementation details of the translation from the specification
language into first-order formula is described in this section.

\todo[inline]{Extend for imports.}
The implementation is organized in the following steps. Fist, the
module is split up into its components. Those are: module description,
programming language, meta-variables, environments, judgments, typing
rules and conjectures. 

\section{Performance}
%%% Local Variables: 
%%% mode: latex
%%% TeX-master: "../report"
%%% End: 
