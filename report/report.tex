% Things to be done before giving away for print
% - spell check
% - let others proof-read

\documentclass[a4paper,twoside]{report}

%% packages
%%%%%%%%%%%
\usepackage[utf8x]{inputenc}
\usepackage[USenglish]{babel}
\usepackage{amsbsy,amscd,amsfonts,amssymb,amstext,amsmath,amsthm,latexsym}
% Package mathpartir is currently not available in nixpkgs
%\usepackage{mathpartir}
\usepackage{stmaryrd}
\usepackage{dot2texi}
\usepackage{tikz}
\usetikzlibrary{shapes,arrows}
\usepackage{url}
\usepackage[xindy]{glossaries}
\usepackage[nottoc]{tocbibind}
\usepackage{pdfpages}

\usepackage{todonotes}

%% options
%%%%%%%%%%
% Currently no such things seem to be needed.
% \newtheorem{definition}{Definition}
% \newtheorem{example}{Example}
% \newtheorem{lemma}{Lemma}
% \newtheorem{theorem}{Theorem}
% \newtheorem{claim}{Claim}
% \numberwithin{definition}{chapter}
\bibliographystyle{alphaurl}
\makeglossaries

%%% Local Variables:
%%% TeX-master: "../thesis"
%%% End:

%% macros
%%%%%%%%%
\newcommand{\loremipsum}{\todo[inline]{Lorem Ipsum}}

%% misc
%%%%%%%%%
\DeclareMathOperator{\pto}{\rightharpoonup}     % partial function arrow
\DeclareMathOperator{\powerset}{\mathcal{P}}    % powerset
\DeclareMathOperator{\proj}{\upharpoonright}    % projection
\DeclareMathOperator{\fst}{\pi_1}               % A x B → A
\DeclareMathOperator{\snd}{\pi_2}               % A x B → B

\def\naive{na\"{\i}ve}
\def\Naive{Na\"{\i}ve}
% Acronyms
%%%%%%%%%%
\newacronym{sdf}{SDF}{Syntax Definition Formalism}
\newacronym{ast}{AST}{Abstract Syntax Tree}
\newacronym{tptp}{TPTP}{Thousands of Problems for Theorem Provers}


%%% Local Variables:
%%% mode: latex
%%% TeX-master: "../thesis"
%%% End:


\begin{document}

% Title page
%%%%%%%%%%%%
\includepdf{titlepage.pdf}

\newpage
\thispagestyle{empty}
\mbox{}

% Erklärung
%%%%%%%%%%%
\input{include/erklaerung}

% Abstract
%%%%%%%%%%
\begin{abstract}
\documentclass{acm_proc_article-sp}

\usepackage[utf8]{inputenc}

\usepackage{cite}

\begin{document}

\title{A Language for the Specification and Efficient Implementation
  of Type Systems}

\numberofauthors{1}

\author{
\alignauthor
Pascal Wittmann
}

\maketitle

\begin{abstract}
  Type systems are important tools to ensure partial correctness of
  programs, to establish abstractions and to guide the programmer in
  the development process. However, there is currently a lack of
  established tools supporting the development of type systems, as
  there are for lexers and parsers. We introduce a declarative
  specification language for type systems, that allows to specify type
  systems in a natural deduction style. We generate from those
  specification two products. A first-order formula representation to
  facilitate the use of automated theorem provers and an efficient
  type checker. Both results aim to make the development cycle for
  type systems faster.
 \end{abstract}

\section{Motivation}
Type systems ensure partial correctness of programs or in other words,
they try to ensure that programs have meaning in the sense of the
semantics of the programming language. The type systems we investigate
are static type systems and can also be thought of as a static
approximation of the program semantics. Besides ensuring partial
correctness, type systems are means to establish abstractions, to
enforce adherence to these abstractions and they can serve as
documentation. All in all type systems can help to develop software
more efficiently (cf. \cite{Petersen:2014:ECS:2597008.2597152} and
\cite{Mayer:2012:ESI:2384616.2384666}).

Type systems are useful tools, if they fit to the programming language
and the application scenario. To ensure that they fit to the
programming language and the application scenario it makes sense to
adapt and modify existing type system or create new specialized type
systems. Those specializations can lead to better error messages, more
expressive type systems and the detection of more
errors, see \cite{Thiemann02programmabletype}. Currently there are, to our
best knowledge, no established tools that generate usable type
checkers from declarative specification languages. Those generators
could make the development of type checkers faster and less error
prone. Those generators would fit well into the language development
workbench besides the long established lexer and parser generators.

The other advantage of generating a type checker from a declarative
specification language is that the specification is close to the
formal description of the type system and the generation of the type
checker can be verified in the absence of a concrete
specification. Often not only the verification of the implementation
is a concern, but also that the type system satisfies certain
properties. As the proofs of propositions about type systems change
when the type system changes, it is desirable to make most of these
changes automatically. Therefore an other motivation for a type
checker generator is to translate specifications into first-order
formulas which can be fed into automated theorem provers to prove
propositions. This establishes a direct connection between the
specified type system, the proofs  and the generated type checker.

\section{Research Problem}
The research problem we are tackling is to create a tool that allows
to create an efficient type checker from a declarative specification
and that reduces the gap between formal reasoning about type systems
and their implementations. This problem splits up in smaller problems.

The first problem is the design of a declarative specification
language that is close to the formalisms used for type systems. This
language should make it easy to use existing syntax definition of
realistic programming languages and the difference between typing
rules on paper and in the specification language should be small.

The next problem is to investigate a good first-order formula
representation of specifications. This representation should be
designed such that it suits automated theorem proving and thus support
a fast prove search. It should also be investigated to which extend
theorem provers for first order logic can be used for type checking.

The main problem is the generation of an efficient type checker. The
performance of the resulting type checker should be comparable with
handcrafted type checkers and the generator should also be able to
cope with non-syntax directed typing rules in an efficient manner. In
this context it should be investigated whether facts proven by an
automated theorem prover can be exploited for the type checker
generation.
\section{Related Work}
There are early approaches for the generation of type checkers that
were not particularly designed for that purpose. For example the
Synthesizer Generator \cite{Reps:1984:SG:800020.808247} which uses
attributes grammars and the ASF+SDF Meta-Environment
\cite{vandenBrand:2001:AMC:647477.727788} based on conditional term
rewriting. Later approaches use similar techniques, for example TCG
\cite{phd/de/Gast2005} which uses inference rules, but are specialized
for the generation of type system. Those approaches have a lower
performance handcrafted implementations, because they implement the
control-flow implied by the used formalism.

A recent approach (TyC \cite{ortin2014automatic}) has focused on the
generation of type checkers for object-oriented programming
languages. It provides a framework to build efficient type checkers
for object-oriented languages (without polymorphism) and uses for this
purpose, in most parts, normal program code that implements an
interface to specify the type system.

Ott \cite{journals/jfp/SewellNOPRSS10} is an other approach that
generates from a specification language code for proof assistants and
\LaTeX, it tries to reduce the gap between the hand-written on-paper
proofs and machine checkable proofs.
\section{Approach}
Our approach tries to combine ideas from the related work presented
above. We design a high level declarative specification language that
is close to formalizations of type systems and generate first-order
formulas and efficient type checkers from it.

We use the language workbench Spoofax \cite{KatsVisser2010} for the
implementation of the specification language and the generators. We
use specifically two parts of Spoofax, the Syntax Definition Formalism
(SDF) to define new syntax and the transformation language Stratego to
generate code from specifications. 

When writing a specification, the syntax of the programming language
must be present in SDF. Specifications are organized in modules and
contain declarations of meta-variables, contexts, judgments and
rules. Figure \ref{fig:example-specification} gives and impression of a
specification.

\begin{figure}
\begin{verbatim}
module example
imports common
language simply-typed-lambda-calculus
meta-variables 	Term "~" { Type Exp }
                Ctx "$" { Context }
                Id "%" { ID }
contexts Context := ID{I} x Type{O}
judgments Context{I} "|-" Exp{I} ":" Type{O}.
rules

%x : ~T in $C
============== T-Var
$C |- %x : ~T

(%x : ~T ; $C) |- ~t : ~T
================================= T-Abs
$C |- \ %x : ~T . ~t : ~T -> ~T

$C |- ~t1 : ~T11 -> ~T12
$C |- ~t2 : ~T2
========================= T-App
$C |- ~t1 ~t2 : ~S
\end{verbatim}
\caption{Specification of the type system for the simply typed lambda
  calculus.}
\label{fig:example-specification}
\end{figure}

A stratego strategy transforms a specification into first-order
formulas in the TPTP\cite{Sutcliffe04tstpdata-exchange} format. The
use of the TPTP format allows to use a variaty of different automated
theorem provers. Typing rules are transformed into roughly into the
following format, where $p_1 \dots p_n$ are the premisses, $c$ is the
conclusion and $free$ collects free variables:

\begin{align}
  \forall v \in free(p_1, \dots, p_n, c) . p_1 \land \dots \land p_n \implies  c 
\end{align}

In this transformation process measurements are taken to ensure that
variables are correctly quantified and valid TPTP code is generated.

Stratego strategies generate the type checker as well. The generated
type checker consists of a constraint generator and a constraint
solver. We have chosen constraint solving as a basis for the type
checker, because it allows the generation of fast type checkers and is
not bound to certain class of type systems. To obtain a fast
constraint generator the structure of the rules should be analyzed,
also with the help of the formula generation and automated theorem
provers. A possible approach for dealing with non-syntax-directed
rules is for instance to check whether two rules commute.
\section{Contributions}
This work has three main contributions. It contributes a declarative
specification language that allows to specify arbitrary type systems
close to mathematical notation. A tool that transforms specifications
into first-order formulas in the TPTP format. Finally a tool that
generates an efficient type checker that can cope with
non-syntax-directed rules and uses automated proves to optimize the
resulting type checker. This reduces the amount of work to implement
type systems and reduces the gap between theory and practice. The
approach is also applicable to type systems from other areas, like
security.

The specification language and the formula generation are nearly
finished and tested with type systems for languages like PCF and
SystemF and with a type system for information flow security. The
generation of the type checker is, by the time of writing, work in
progress.
\bibliography{bibliography}{} \bibliographystyle{plain}
\end{document}

\end{abstract}

% Table of contents
%%%%%%%%%%%%%%%%%%%
\tableofcontents

\todo[inline]{Where do I put the "I would like to thank …"}
\todo[inline]{Search for a better word for 'goal'}

% Content
%%%%%%%%%
\chapter{Introduction}
\section{Motiviation}
\todo[inline]{Somehow mention that we are always talking about static
  type systems}

Type systems try to solve the problem of ensuring that programs have
meaning. They assign types to expressions and have rules that state
which expressions of which types can be combined. A type checker
ensures that the expressions of a program meet those rules, if those
rules are met, the program is called well-typed. Essentially a type
system ensures that a program has meaning, by approximating its
semantics. They also can be used to establish abstractions and to
enforce the adherence to these abstractions.


\todo[inline]{Extract everything useful from this text and
build a readable motivation from it}

Type systems are means to put focus on meaningful programs. In other
words they help the programmer to catch errors by helping building
abstractions and by enforcing those abstractions thorough the
program. We are dealing with static type systems, i.e. abstractions
are enforced before the execution. Those type systems help to reduce
development time and to reveal errors earlier. \todo{Cite Stefan
  Hanenberg}

Static type systems are typically integrated into the compiler of the
programming language. \todo{Proof this} This makes it hard to extend a
given type system or to implement a type system for an (embedded)
domain specific language. In addition the implementation of a type
system is time expensive and error prone.

The goal of this thesis is to design a language for the specification
of static type systems in the style of natural deduction and to
generate based on such specifications an efficient implementation.

\section{Contributions}
The main contributions of this thesis are:
\begin{enumerate}
\item A declarative specification language for static type systems, in
  which typing rules can be written in the style of natural deduction
  and which builds SDF definitions of programming languages.
\item The transformation of type system specifications into equivalent
  first-order formulas in the TPTP format, suitable to be passed to
  automated theorem provers.
\item A generator that creates type checkers from type system
  specifications and which is designed for performance.
\end{enumerate}

\section{Structure}
\todo[inline]{Describe how the thesis is structured.}

\chapter{Preliminaries}
\section{Tools}
\subsection{Spoofax}
\subsection{SDF}
\subsection{Stratego}
\subsection{Sugar-*}
\subsection{Alternatives}
\section{Type Systems}
\textit{Type theory} started as an attempt by Gottlob Frege to solve Russel's
paradox, that shows that \naive set theory is inconsistent. Frege
argued that a predicate requires an object as argument and cannot have
itself as an argument, as it is the subject.\todo{Make this more
  precise} So the initial motivation for type theory was to avoid
paradoxes and contradictions in logics and rewrite systems. The term
\textit{type system} refers to type theories whose logics rewrite
systems are programming languages. Type systems address therefore
similar problems as general type theory. The problem they address is
to ensure that programs have meaning, whereas in type theory the
problem is the ensure the consistency of a logic.

What does it mean to ``ensure that programs have meaning''. First, it
means that one wants to filter the useful programs. It is not useful
to have a syntactically valid program that has no semantics. This is
done by assigning types to the expressions of the program. A
\textit{type checker} can then check, whether those types match with
the expressions. If the type checker succeeds the program is
\textit{well-typed} and has meaning. Second, a type system allows to
establish abstractions through user definable data types. When type
checking a program, it is also ensured that those abstractions are not
violated.\footnote{This is only true for so called ``strong'' type
  systems.} In this thesis \textit{static} type systems are
considered. In a static type system types are checked prior to
execution. This early checks can provide guarantees about the programs
behavior 

\section{First-Order Logic}


\chapter{Specification Language}
\section{Goals}
\section{Design \& Architecture}
\section{Implementation}

\chapter{Formula Generation}
\section{Goals}
\section{Why First-Order Logic?}
\subsection{Expressiveness}
\subsection{Tool-Support}
\section{Translations}
\section{Implementation}
\section{Performance}

\chapter{Type Checker Generation}
\todo[inline]{Write a section for each approach that was attempted}
\section{Performance}

\chapter{Applications}
\section{Integration in Tool Chains}
\section{Security Type Systems}

\chapter{Summary}
\section{Conclusion}
\section{Future Work}
\section{Related Work}
\subsection{Specification Languages}
\subsection{Type Checker Generation}

% Bibliography
%%%%%%%%%%%%%%
\bibliography{bibliography}

\end{document}
