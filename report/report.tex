% Things to be done before giving away for print
% - spell check
% - let others proof-read

\documentclass[a4paper,twoside]{report}

%% packages
%%%%%%%%%%%
\usepackage[utf8x]{inputenc}
\usepackage[USenglish]{babel}
\usepackage{amsbsy,amscd,amsfonts,amssymb,amstext,amsmath,amsthm,latexsym}
% Package mathpartir is currently not available in nixpkgs
%\usepackage{mathpartir}
\usepackage{stmaryrd}
\usepackage{dot2texi}
\usepackage{tikz}
\usetikzlibrary{shapes,arrows}
\usepackage{url}
\usepackage[xindy]{glossaries}
\usepackage[nottoc]{tocbibind}
\usepackage{pdfpages}

\usepackage{todonotes}

%% options
%%%%%%%%%%
% Currently no such things seem to be needed.
% \newtheorem{definition}{Definition}
% \newtheorem{example}{Example}
% \newtheorem{lemma}{Lemma}
% \newtheorem{theorem}{Theorem}
% \newtheorem{claim}{Claim}
% \numberwithin{definition}{chapter}
\bibliographystyle{alphaurl}
\makeglossaries

%%% Local Variables:
%%% TeX-master: "../thesis"
%%% End:

%% macros
%%%%%%%%%
\newcommand{\loremipsum}{\todo[inline]{Lorem Ipsum}}

%% misc
%%%%%%%%%
\DeclareMathOperator{\pto}{\rightharpoonup}     % partial function arrow
\DeclareMathOperator{\powerset}{\mathcal{P}}    % powerset
\DeclareMathOperator{\proj}{\upharpoonright}    % projection
\DeclareMathOperator{\fst}{\pi_1}               % A x B → A
\DeclareMathOperator{\snd}{\pi_2}               % A x B → B

% Acronyms
%%%%%%%%%%
\newacronym{sdf}{SDF}{Syntax Definition Formalism}
\newacronym{ast}{AST}{Abstract Syntax Tree}
\newacronym{tptp}{TPTP}{Thousands of Problems for Theorem Provers}


%%% Local Variables:
%%% mode: latex
%%% TeX-master: "../thesis"
%%% End:


\begin{document}

% Title page
%%%%%%%%%%%%
\includepdf{titlepage.pdf}

\newpage
\thispagestyle{empty}
\mbox{}

% Erklärung
%%%%%%%%%%%
\input{include/erklaerung}

% Abstract
%%%%%%%%%%
\begin{abstract}
\documentclass{acm_proc_article-sp}

\usepackage[utf8]{inputenc}

\usepackage{cite}

\begin{document}

\title{A Language for the Specification and Efficient Implementation
  of Type Systems}

\numberofauthors{1}

\author{
\alignauthor
Pascal Wittmann
}

\maketitle

\begin{abstract}
  Type systems are important tools to ensure partial correctness of
  programs, to establish abstractions and to guide the programmer in
  the development process. However, there is currently a lack of
  established tools supporting the development of type systems, as
  there are for lexers and parsers. We introduce a declarative
  specification language for type systems, that allows to type systems
  in a natural deduction style. Two results are generated from those
  specifications. A first-order formula representation to facilitate
  the use of automated theorem provers and an efficient type
  checker. Both results aim to support a faster development cycle for
  type systems.
 \end{abstract}

\section{Motivation}
Type systems ensure partial correctness of programs or in other words,
they try to ensure that programs have meaning in the sense of the
semantics of the programming language. The type systems we investigate
are static type systems and can also be thought of as a static
approximation of the program semantics. Besides ensuring partial
correctness, type systems are means to establish abstractions and to
enforce adherence to these abstractions. All in all this can help to
develop software more efficiently
(cf. \cite{Petersen:2014:ECS:2597008.2597152} and
\cite{Mayer:2012:ESI:2384616.2384666}).

Type systems are useful tools, if they fit to the programming language
and their use. To ensure that they fit to the programming language and
their use it makes sense to adapt and modify existing type system or
create new specialized type systems. Those specializations can lead to
better error messages, more expressive type systems and the detection
of more errors. Currently there are, to our best knowledge, no
established tools that generate usable type checkers from declarative
specification languages. Those generators could make the development
of type checkers faster and less error prone. Those generators would
fit well into the language development workbench besides the long
established lexer and parser generators.

The other advantage of generating a type checker from a declarative
specification language is that the specification is close to the
formal description of the type system and the generation of the type
checker can be verified in the absence of a concrete
specification. Often not only the verification of the implementation
is a concern, but also that the type system satisfies certain
properties. As the proofs of propositions about type systems change
when the type system changes, it is desirable to make most of these
changes automatically. Therefore an other motivation for a type
checker generator is to translate specifications into first-order
formulas which can be fed into automated theorem provers to prove
propositions. This establishes a direct connection between the
propositions and the generated type checker.

%% TODO: Mention the application to other kind of type systems?

\section{Research Problem}
The research problem we are trying to tackle is to create a tool that
allows to create an efficient type checker from a declarative
specification and that reduces the gap between formal reasoning about
type systems and their implementation. This problem splits up in
several smaller problems.

The first problem is the design of a declarative specification
language that is close to the formalisms used for type systems. This
language should make it easy to use existing syntax definition of
realistic programming languages and the difference between typing
rules on paper and in the specification language should be small.

The next problem is to investigate a good first-order formula
representation of specification. This representation should be
designed such that it suits automated theorem proving and thus fast
proves. It should also be investigated to which extend theorem provers
for first order logic can be used for type checking.

The main problem is the generation of an efficient type checker. The
performance of the resulting type checker should be comparable with
type checkers that are implemented by hand. The generator should also
be able to cope with non-syntax directed typing rules in an efficient
manner. As one way to achieve this it should be investigated whether
facts proven by an automated theorem prover can be exploited.
\section{Related Work}
\section{Results}
\section{Contributions}

\end{document}
\end{abstract}

% Table of contents
%%%%%%%%%%%%%%%%%%%
\tableofcontents

\todo[inline]{Where do I put the "I would like to thank …"}
\todo[inline]{Search for a better word for 'goal'}
\todo[inline]{How do we refer to all the different types of
  languages? This should be done uniformly.}

% Content
%%%%%%%%%
\chapter{Introduction}
\section{Motiviation}
\todo[inline]{Somehow mention that we are always talking about static
  type systems}
\todo[inline]{Mention studies on type systems, e.g. by Hanenberg}

Type systems try to solve the problem of ensuring that programs have
meaning. They assign types to expressions and have rules that state
which expressions of which types can be combined. A type checker
ensures that the expressions of a program meet those rules, if those
rules are met, the program is called well-typed. Essentially a type
system ensures that a program has meaning, by approximating its
semantics. They also can be used to establish abstractions and to
enforce the adherence to these abstractions.

Although type systems are a useful tool increase software quality,
there is a lack of tools that help to create and modify type
checkers. Almost all type checkers are implemented by hand and are far
away from their formal models. In addition it is non-trivial to write
or even extend a type checker. Therefore current tools are not suited
for quick prototyping of type system features for new or existing
languages. This is also true for the development of domain specific
languages (DSL). In that sense type systems are underprivileged in
comparison to parsers. Parser generators are widely spread and
accepted tools.

The goal of this thesis is to close that gap in the tool chain. A
declarative specification language is developed, in which type systems
can be represented very close to the standard formalisms. Two things
are generated from those specifications. First, a representation of
the type system as first-order formulas, intended to support proving
properties of the type system. Second, an efficient type checker that
can be integrated in existing tool chains.
\section{Contributions}
The main contributions of this thesis are:
\begin{enumerate}
\item A declarative specification language for static type systems, in
  which typing rules can be written in the style of natural deduction
  and which builds SDF definitions of programming languages.
\item The transformation of type system specifications into equivalent
  first-order formulas in the TPTP format, suitable to be passed to
  automated theorem provers.
\item A generator that creates type checkers from type system
  specifications and which is designed for performance.
\end{enumerate}

\section{Structure}
\todo[inline]{Describe how the thesis is structured.}

\chapter{Preliminaries}
\section{Tools}
\subsection{Spoofax}
\subsection{SDF}
\subsection{Stratego}
\subsection{Sugar-*}
\subsection{Alternatives}
\section{Type Systems}
\textit{Type theory} started as an attempt by Gottlob Frege to solve Russel's
paradox, that shows that \naive set theory is inconsistent. Frege
argued that a predicate requires an object as argument and cannot have
itself as an argument, as it is the subject.\todo{Make this more
  precise} So the initial motivation for type theory was to avoid
paradoxes and contradictions in logics and rewrite systems. The term
\textit{type system} refers to type theories whose logics rewrite
systems are programming languages. Type systems address therefore
similar problems as general type theory. The problem they address is
to ensure that programs have meaning, whereas in type theory the
problem is the ensure the consistency of a logic.

What does it mean to ``ensure that programs have meaning''. First, it
means that one wants to filter the useful programs. It is not useful
to have a syntactically valid program that has no semantics. This is
done by assigning types to the expressions of the program. A
\textit{type checker} can then check, whether those types match with
the expressions. If the type checker succeeds the program is
\textit{well-typed} and has meaning. Second, a type system allows to
establish abstractions through user definable data types. When type
checking a program, it is also ensured that those abstractions are not
violated.\footnote{This is only true for so called ``strong'' type
  systems.} In this thesis \textit{static} type systems are
considered. In a static type system types are checked prior to
execution. This early checks can provide guarantees about the programs
behavior 

\section{First-Order Logic}


\chapter{Specification Language}
\section{Goals}
\section{Design \& Architecture}
\section{Implementation}

\chapter{Formula Generation}
\section{Goals}
\section{Why First-Order Logic?}
\subsection{Expressiveness}
\subsection{Tool-Support}
\section{Translations}
\section{Implementation}
\section{Performance}

\chapter{Type Checker Generation}
\todo[inline]{Write a section for each approach that was attempted}
\section{Performance}

\chapter{Applications}
\section{Integration in Tool Chains}
\section{Security Type Systems}

\chapter{Summary}
\section{Conclusion}
\section{Future Work}
\section{Related Work}
\subsection{Specification Languages}
\subsection{Type Checker Generation}

% Bibliography
%%%%%%%%%%%%%%
\bibliography{bibliography}

\end{document}
